\subsection{Information Flow}
Our proposed work on decentralized network design will ensure that the
underlying communication network connecting the agent nodes satisfies
mission-critical connectivity properties.  The network structures thus
designed are very likely to change with time due to the mobility of
the nodes, failures in the communication links, as well as adversarial
attacks on the network.  

The second major component of this proposal to develop distributed
algorithms for efficient and effective flow of information over such
dynamic networks.  Our proposed work on network design allows us to
make assumptions about minimal connectivity of the underlying network;
however, to guarantee high-throughput information flow for a dynamic
distributed environment, we need to control congestion, and focus on
local lightweight algorithms that maintain limited state.  In this
project, we will study two important information flow problems:
congestion-minimizing degree-bounded flows, and information
dissemination under adversarial dynamics.

\subsubsection{Congestion-minimizing degree-bounded flows in dynamic
  networks} In earlier work \cite{awerbuch+l:flow}, distributed
algorithms for finding congestion minimizing flows in certain dynamic
networks were presented.  However the routes undertaken by these flows
were unconstrained and expensive to maintain.  We plan to build on our
past work on efficient algorithms for converting optimal flows into
near-optimal degree-bounded
flows~\cite{chen+klrsv:confluent,chen+smr:confluent}; our current
algorithms involve some refinement techniques that require global
knowledge.  We believe that the local balancing approach
of~\cite{awerbuch+l:flow} in conjunction with our flow refinement
techniques will enable us to maintain near-optimal degree-bounded
flows in dynamic networks.  We also plan to consider generalizations
of the problem incorporating link errors and coding techniques.

This problem can be generalized in a multiplicity of ways. Rather than
one commodity one can consider multiple commodities, each with their
own target or destination nodes. Second, one can consider the
availability of end-to-end capacities assuming duplex
connections. Third, one can assume different forms of error that
affect link capacities and the use of techniques such as network
coding. Finally, one can also consider integrating the route discovery
problem with the problem of optimizing the flows.

\subsubsection{Information dissemination in adversarial networks} 
We will conduct a comprehensive study of fully-distributed gossip
algorithms for the $k$-broadcast problem in dynamic networks.  Recall
that in the $k$-broadcast problem, $k$ of the $n$ nodes each have a
message that need to be disseminated to every node in the network.
Suppose that the nodes are synchronized and in each step, each node
can broadcast the equivalent of a bounded number of tokens to its
neighbors~\cite{kuhn+lo:dynamic}.  What is the minimum number of steps
needed to complete the dissemination?  If the network is completely
static and connected, then a local token-forwarding process on a
spanning tree of the network can accomplish the task in $O(n + k)$
steps, independent of the structure of the network.  In a dynamic
network as in the above model however, the problem is much more
challenging.

In recent work~\cite{dutta+prs:dynamic}, we studied the class of {\em
  forwarding}\/ algorithms that do not manipulate tokens in any way
other than copying, storing, and forwarding them.  We show that {\em
  any}\/ forwarding algorithm will take $\Omega(nk/\log(n))$ steps to
complete $k$-broadcast, thus resolving an open problem
of~\cite{kuhn+lo:dynamic}.  Given that almost any local greedy
forwarding procedure completes $k$-broadcast in $O(nk)$ steps in any
dynamic network, our lower bound essentially captures the severe
limitations imposed by highly adversarial network dynamics.

A natural and attractive alternative to forwarding algorithms is to
use coding (either end-to-end~\cite{Byers02adigital,Shok06} or
network~\cite{ahlswede+cly:coding}).  Recent
work~\cite{haeupler:gossip,haeupler+k:dynamic} has shown that
information spreading based on network coding can solve $k$-broadcast
in $O(n+k)$ steps, assuming the sizes of the messages are $\Omega(n
\log n)$ bits.  While this message size lower bound is prohibitively
large and impractical (since it scales with the size of the network),
our lower bound~\cite{dutta+prs:dynamic} together with this upper
bound establish, in theory, a fundamental gap between flow-based and
coding-based dissemination procedures.

One approach we plan to pursue is to consider a hybrid
forwarding-coding algorithm in which nodes exchange information in the
symmetric difference of what they currently hold, which can be done in
$O(\log n)$ rounds of communication using
fingerprinting~\cite{mitzenmacher-2005-fastmixing}.  We have shown 
that if the entropy of the initial distribution of information
is high, then convergence to full dissemination is rapid.  We will
also consider weaker notions of the adversary (e.g., offline or
oblivious), which model real-world settings where the adversary has
significant control but is not congnizant of all the network actions.
The offline dissemination problem can be reduced to the problem of
packing Steiner trees in directed
graphs~\cite{cheriyan+s:steiner,dutta+prs:dynamic}, and thus has deep
connections with the long-standing open problems of approximating
directed Steiner
trees~\cite{charikar+ccdgg:steiner,zosin+k:steiner,halperin+k:steiner}
and bounding the network coding advantage in multicast over directed
networks~\cite{agarwal+c:coding,sanders+et:flow}.

We expect to quantify our results in terms of relevant parameters
including locality of the dynamics, conductance/expansion of the
evolving network, initial entropy of the information distribution, and
knowledge available to the dissemination algorithm.  We have also
shown that the dissemination problem can be reduced to the problem of
packing Steiner trees in directed graphs, and thus has deep
connections with the long-standing open problems of approximating
directed Steiner trees~\cite{charikar+ccdgg:steiner} and bounding the
network coding advantage in multicast over directed networks.

