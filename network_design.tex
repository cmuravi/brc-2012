\subsection{Network Design}
\subsubsection{Survivable Network Design}  
While the survivable network design problem is known to admit good
approximation algorithms in the offline case, work in the online
setting has mainly focused only on one-connected problems such as
Steiner and generalized Steiner trees~\cite{qw11}. Indeed, no online
algorithms were known for higher connected versions of the problem
until our recent work~\cite{gkr10}. Our algorithms rely on certain
tree embeddings and are inherently centralized.  We propose to study
decentralized algorithms for survivable network design in an online
setting where the underlying graph and costs are static while the
connectivity requirements change online.

\subsubsection{Aggregation trees} 
Consider the problem of building an aggregation tree for tracking
targets.  In a stochastic model of uncertainty, we may view each
target as an object that is detected by one of the various agents
according to a previously determined probability distribution.  In
order to build a backbone network of the graph for coordination and
control, we need to determine a subtree of the graph containing the
root such that given the probability distribution of target
materialization, the subtree maximizes the expected number of targets
that can be covered by the agent nodes in the tree. If this backbone
tree is a spanning tree, then all targets (each of which materializes
at any one of the nodes) will be covered. At the other end, if no
backbone or tree edges can be built a priori, the root can only detect
targets that materialize in its vicinity. We thus have a trade-off
between the allowed size of the Steiner tree and its expected target
coverage that we propose to study in detail.

We call this problem the \noindent{\bf \em Target Maximizing Rooted
  Steiner Tree} problem.  We are given an undirected graph $G =
(V,E)$, a root node $r$, and a set of targets $T_i, i = 1 \ldots
t$. Each target $T_i$ is a probability distribution $p_i$ over $V$
where $p_i(v)$ is the probability that the target $i$ is at node $v$
(Note that $p_i(v) \geq 0 \ \forall v$ and $\sum_{v \in V} p_i(v) = 1
\ \forall i$). Assume that the distributions $p_i$ and $p_j$ for $i
\neq j$ are independent of each other. Given a target size $S$, our
goal is to find a tree $F$ containing the root of at most $S$ edges
such that the expected size of the targets that materialize in the
node set of the tree is maximized.

This problem can be generalized in various ways. First, the size bound
$S$ can be extended to take into account distances between nodes or
the relative costs of establishing connection between these pairs of
agents. Second, the objective function can be changed to reflect other
notions of reward such as the probability of detecting at least $N$
targets for a given $N$. Third, the independence assumption on target
materialization can be relaxed; this would not make any difference for
the expected target size function but is interesting for other
objectives. Fourth, the tree construction process can be modeled as
the vehicle tour of a control packet in the ad hoc network, leading to
vehicle routing problems that try to minimize expected size to cover
all targets say. Fifth, the notion of stochastic uncertainty can be
replaced by the requirement that the method is robust towards the
worst of many possible scenarios of target materialization. We
investigate one particularly strong notion of robustness for this
problem next.

\subsubsection{Universal approximations for Steiner trees} 
We also plan to study aggregation trees under an adversarial model of
uncertainty.  We have introduced the framework of {\em universal
  approximations}\/ that provides a robust notion of quality with
respect to {\em any} online sequence of
arrivals~\cite{jia+lnrs:universal}.  Universality is a framework for
dealing with uncertainty by guaranteeing a certain quality of goodness
for all possible completions of the partial information set.
Formally, an instance of the Universal Steiner Tree (\ust) problem is
a pair $\langle G , r\rangle$ where $G = (V,E)$ is a weighted
undirected graph, and $r$ is a distinguished vertex in $V$ that we
refer to as the {\em root}.  For any spanning tree $T$ of $G$, define
the {\em stretch}\/ of $T$ as $\max_{S \subseteq V}\cost{T_{S\cup
    r}}/\cost{\OptSt{S\cup r}}$, where $\OptSt{X}$ is an optimal
Steiner tree connecting the vertices in $X \cup r$.  The goal of the
\ust\ problem is to determine a spanning tree with minimum stretch.

Similar to the \ust\ problem, we can define universal versions of
Traveling Salesperson Problem (\utsp), group Steiner tree, group TSP,
and generalized Steiner network problems.  An instance of \utsp\ is a
metric space $(V,d)$.  For any cycle (tour) $\TR$ containing all the
vertices in $V$ and a subset $S$ of $V$, let $\TR_S$ denote the unique
cycle over $S$ in which the ordering of vertices in $S$ is consistent
with their ordering in $C$.  The stretch of $\TR$ is defined as
$\max_{S\subseteq V} \cost{\TR_{S}}/\cost{\Optr{S}}$, where $\Optr{S}$
denotes the minimum cost tour on set $S$. The \utsp\ is to find a tour
on $V$ with minimum stretch.

\Research{
\label{prob:ust}
What are the best stretch achievable for \ust, \utsp, and universal
variants of the group Steiner tree, group TSP, and generalized Steiner
network problems?}

The notion of universality is captured by the complexity class
$\Sigma_2^P$.  Previous work has shown that for any metric space,
there exist Steiner trees that have universal poylogarithmic
approximation ratios: our prior results~\cite{jia+lnrs:universal}, and
the improved results of $O(\log^2 n)$~\cite{gupta+hr:oblivious} that
introduce the notion of oblivious network design.  Though these
results reveal new onsights to the structural properties of Steiner
trees and metrics, a major drawback is that {\em they do not apply to
  arbitrary graphs}.  

In very recent work, we have made some progress on this
front~\cite{busch+drrs:ust}, and have identified a promising approach
to attack the problem.  One of the major challenges in constructing a
universal Steiner tree is that any tree will be forced to place some
vertices far apart in tree, even though they may be ``nearby''
according to the underlying graph distances.  As a result, the
resulting spanning tree may perform poorly on a subset that includes
this set and some other carefully chosen vertices.  To address this
challenge, we introduce the notion of a {\em hierarchy of graph
  partitions}, each of which guarantees small strong cluster diameter
and bounded local neighbourhood intersections.  We have shown that the
such a suitable hierarchy of graph partitions is essentially both
sufficient and necessary for constructing low-stretch universal
Steiner trees.  For metric spaces -- i.e., weighted complete graphs
satisfying triangle inequality -- such a hierarchy can be constructed
using the seminal work of Awerbuch and Peleg on sparse
partitions~\cite{awerbuch+p:partition,peleg:distributeBook}.  

It is a major open problem, however, to build such hierarchies for
arbitrary graphs.  We have made preliminary progress by presenting
partition hierarchies for general graphs that yield a $2^{O(\sqrt{\log
    n})}$-stretch \ust\ for general graphs, and partition hierarchies
for minor-free graphs that yield a polylogarithmic-stretch \ust\ for
minor-free graphs.  Furthermore, all of the solutions proposed thus
far are centralized.  We propose to develop distributed algorithms for
constructing sparse partition hierarchies.  In an earlier work on
aggregation trees for sensor network applications, we showed that this
is achievable for the highly specialized case of grid
graphs~\cite{jia+nrs:gist}.

\iffalse

RELATED WORK: karger-minkoff maybecast tree, gupta-nagarajan-ravi
paper on adaptive TSP (icalp 10), and on stochastic vrp (OR, 2012) and
refs in this for previous papers on TSP with independent demands on
nodes (Bertsimas cycle heuristic). Also the basic expected target max
is like orienteering with size bound on tree, and hence related to
k-MST. Also related to garg-gupta-leonardi-sankowsi (SODA 08) which
relates to Univ TSP.
\fi

