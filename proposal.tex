\documentclass{article}
%\usepackage[rflt]{floatflt}
\usepackage{wrapfig}
\usepackage{subfigure}
\usepackage{latexsym}
\usepackage{amssymb}
\usepackage{color}
\usepackage{multirow}
\usepackage{arydshln}

\usepackage{times}

\renewcommand{\baselinestretch}{1.00}
\newcommand{\hide}[1]{}
\newcommand{\red}[1]{\textcolor{red}{#1}}
\newcommand{\blue}[1]{\textcolor{blue}{#1}}
\newcommand{\listspace}{\setlength{\itemsep}{0pt}\setlength{\parskip}{2pt}\setlength{\topsep}{0pt}}

\oddsidemargin = -0.13in
\evensidemargin = -0.13in
\textwidth = 6.72in
\topmargin = -0.76in
\textheight= 9.4in
\setlength{\textfloatsep}{0pt}

\newcommand{\BfPara}[1]{{\noindent {\bf #1.}}}
\newcommand{\junk}[1]{}

\title{White Paper for 2012 Research Opportunity:\\
Decentralized Online Optimization\\
\ \\
Provably Near-Optimal Distributed Online Network Optimization\\}
%Algorithmic frameworks for uncertainty and decentralization\\}
%{\em R. Ravi (CMU), Rajmohan Rajaraman \& Ravi Sundram (Northeastern U.)}}

\author{Technical Point of Contact: Prof. R. Ravi\\
 Tel: +1 412 268 3694\\
 Fax: +1 412 268 7345\\
 Email: ravi@cmu.edu}
 %(CMU), Rajmohan Rajaraman \& Ravi Sundram (Northeastern U.)}

\date{March 29, 2012}
\begin{document}

%\thispagestyle{empty}

\maketitle
\titlepage

\iffalse***************
Background:

The Navy is moving towards deploying large, complex systems that are beyond centralized
control. A canonical example of such a system is a fleet of unmanned vehicles with limited
communications operating in a dynamic environment.  Important characteristics of these systems
are that 1) they are decentralized (i.e., system components can take independent actions), and 2)
the environment in which the system operates is not necessarily known a priori, and is revealed
over time; that is, the data defining the system and its environment is online (in the sense of
online algorithms).

Objective:

The objective of this topic is to develop scientific principles and algorithms for solving
decentralized, online optimization problems.  To achieve this, first, solid mathematical
frameworks need to be proposed and put into place so that various algorithmic strategies can be
developed, analyzed, and compared.  Second, canonical models need to be defined.  These
models should capture the fundamental difficulties associated with decentralized, online
optimization.  The aim in defining a few, simple canonical models is not to include all possible
real-world complexities, but rather create a set of models whose rigorous treatment will drive
design and analysis principles.  Third, promising algorithmic strategies need to be identified and
developed.

This Basic Research Challenge (BRC) topic seeks proposals that scientifically
address efficient, robust computational techniques having the following characteristics:

1) The techniques can be applied to problems possessing a high degree of decentralization,
and in which there is limited communication between system components,

2) The techniques can be applied to problems in which not all relevant information of the
environment is known a priori, and is revealed incrementally to individual system
components; information spreads through the system as communications become
available and with potentially substantial time delays, and

3) The techniques are capable of producing high-quality solutions in a reasonable amount of
time.  Solution quality is measured against optimality, and solution time is measured
against the time scale of changes in the environment.  Analysis of these measures is
expected to be mathematically rigorous.

Only proposals that meet all three criteria will be considered within the scope of this BRC topic.


---------------------------



Battlefield locations show that information is revealed online. By coordinating, better solutions can be found?

E.g. Adaptive optima versus non adaptive optimum in the k-center case. Locate ships to maximize detecting targets.

Point to algorithmic game theory as a motivation to develop a theory that quantifies the price of decentralization.

Build a basic model with agents as computational decision making units with limited communication. At one end, with very little communication, what can be achieved w.r.t. optimal, and at the other, what is the best performance guarantee? E.g., choosing an independent set in a distributed system. Or maybe choosing a distributed minimum spanning tree?


---------------------

We motivate the problem of distributed decentralized optimization in
the
context of wireless communication where nodes are distributed
spatially and
have individual and coalitional utilities as in a cognitive radio or
anti-jam context. We then say this problem is complex as it involves
dealing with both nature and adversary in an online fashion and with
distributed information sets.
So we will explore different facets of this complex scenario by
restricting
certain aspects or simplifying certain issues. This will allow us to
talk about universal problems and then move on to stochastic and
robust optimization as ways to deal with future uncertainty. We will
then move onto distributed network
formation and speculate about problems in the game-theoretic context.

----------------------

White papers should not exceed four (4) single-sided pages, exclusive of cover page and resume of principal investigator, and should be in 12-point Times New Roman font with margins not less than one inch. The cover page should be labeled �White Paper for 2012 Research Opportunity: �Decentralized Online Optimization� and include the following information: title of the proposed effort, technical point of contact, telephone number, fax numbers, and e-mail address. The four (4) page body of the white paper should include the following information: (1) Principal Investigator; (2) Relevance of the proposed effort to the research areas described in Section II; (3) Technical objective of the proposed effort; (4) Technical approach that will be pursued to meet the objective; (5) A summary of recent relevant technical breakthroughs; and (6) A funding plan showing requested funding per fiscal year. A resume of the principal investigator, not to exceed one (1) page, should also be included after the four (4) page body of the white paper.

\fi%%%%%%%%%%%%%%%%%%%%%%%%%%%

%\begin{abstract}

%\end{abstract}

\section{PI Information}
Prof. R. Ravi works in the intersection of Operations Research and
Computer Science and has pioneered work in various network
optimization problems such as the Buy-at-Bulk Network Design problem
with economies of scale, the Group Steiner Tree problem that combines
the Set Covering and Steiner Tree problems, and in designing provably
near-optimal approximation algorithms with performance guarantees for
problems in stochastic and robust combinatorial optimization. He has
also advanced the development of tools for the design of approximation
algorithms for network optimization such as the primal-dual method,
dependent randomized rounding, Lagrangean relaxations and iterative
methods.  

Co-PI Rajaraman's research expertise covers distributed computing
theory, approximation algorithms, network optimization, and
algorithmic game theory.  His work on distributed hash tables is
widely-cited and has been implemented in several peer-to-peer network
systems.  Co-PI Sundaram's research expertise covers networks,
algorithms, complexity theory and combinatorics.  Before joining
academics, he was Director of Engineering at Akamai Technologies,
where he established the mapping group for the world's leading content
delivery network, which is responsible for directing browser requests
(over 10 billion a day) to the optimal Akamai server.

The PIs have a strong history of collaboration.  PI Ravi and co-PI
Sundaram were co-authors on a widely cited foundational paper on
bicriteria approximations in network design.  Co-PIs Rajaraman and
Sundaram were co-authors on a ICDCS 2006 paper on Internet capacity
that won the best paper award.  With deep synergistic expertise in
optimization theory, online and approximation algorithms, and
distributed computing, the team is uniquely positioned to address the
basic research challenges of this call.

\section{Relevance}
% Relevance of the proposed effort to the research areas described in Section II

%\section{Motivation}
As communication devices and sensor infrastructures scale at a rapid
rate, it has now become possible to deploy large ad-hoc networks that
communicate in a limited fashion without a centralized control, and
carry out useful tasks (such as surveillance and threat detection)
even with limited coordination and control. This naturally leads us to the consideration of {\em distributed} algorithms that use little or no central control.

The lack of control is further exacerbated by the lack of complete
information about the environment in which the agents are located,
adding an extra layer of uncertainty to the problem. The paucity of
information can be modeled either by providing a probabilistic model
of the information which leads to {\em stochastic optimization}
models~\cite{bl97,rs06}, or by an adversarial model of information
revelation that leads to competitive analysis in the sense of {\em online
optimization} models~\cite{be98}. When the scenarios of uncertainty are neither quantifiable using randomness, nor adversarial, the framework of {\em robust optimization} models~\cite{dgrs05} is also useful.

In this proposal, we develop robust models for studying the
fundamental tasks of establishing and maintaining connectivity and
control in complex networked systems, which capture (i) limited
communication between agents (necessitating distributed algorithms),
(ii) online revelation of information over time (including stochastic,
robust and competitive frameworks) and (iii) permit the development of
rigorous approximation algorithms with a polynomial running time that
provide provably good solutions with performance guarantees. These
three aspects of our model directly address the three main
characteristics in this BRC topic.

\section{Technical Goals}
%\section{Models}
\label{sec:goals}

The main goal of this project is to develop a comprehensive theory of
{\bf \em distributed online approximation}\/ algorithms for hard network
optimization problems.  
\iffalse (Repeats what is in the prev para)
This proposal encompasses several challenging
aspects of mission-critical networked systems: (a) there is
considerable uncertainty in the inputs and the network environment
under which the systems operate; (b) the inputs as well as the
underlying network may, in fact, be under adversarial control; (c) the
algorithms running these networked systems need to be
fully-distributed.
\fi
At a high level, the problems we plan to study in this project can be
divided into two categories: {\bf \em network design}\/ and {\bf \em
information flow}.  

\subsection{Network Design}
Network design concerns the construction of overlay network structures
that form the foundation for aggregation, point-to-point routing,
broadcast, multicast and other critical network functions.  The
deployment of mission-critical military systems requires the ability
to construct and maintain such large-scale network structures that
will enable secure and reliable communication and operation in a
highly dynamic and distributed envivornment.  Research in network
design has been at the forefront of major advances in approximation
algorithms.  We believe that theoretical foundations of distributed
online network design are essential to achieve major advances in this
area.  Within network design, we will study the general problem of
{\em survivable network design}, with a focus on the important special
case of {\em aggregation trees}.

\smallskip
\BfPara{Survivable network design} In the {\em survivable network design problem}, we are given a graph $G = (V,E)$ with edge-costs, and edge-connectivity requirements $r_{ij} \in Z_{\ge 0}$ for every pair of vertices $i, j \in V$, and need to find an
(approximately) minimum-cost network that provides the required
connectivity.  This is one of the most fundamental problems in network
design that generalizes several graph-theoretic optimization problems including shortest paths, spanning and Steiner trees.
The edge-connectivity requirements capture the need for increased
reliability in mission-critical systems; furthermore, the general
statement of the problem allows one to develop algorithmic paradigms
that may have broad applicability.  

\smallskip
\BfPara{Aggregation trees} A
special case of survivable network design is the fundamental problem
of constructing a tree that aggregates information from important
agents in a distributed network to a central point (HQ). The agents
are modeled as nodes in a graph and the limited communication pattern
between agents are represented as edges of the graph. Suppose the
primary task of the network is to detect and monitor specific targets
whose location the agents are unaware of a priori.  As the agents
explore their local terrain in detail they may become aware of the
presence of valuable targets. This converts such a node into a {\em
terminal} node that must now communicate with the HQ node henceforth
called the {\em root}. This problem is a variant of the classical
minimum hop Steiner tree problem. If we assume that the graph
connectivity pattern does not change for simplification, the extra
complication is that the location of the terminals are unknown.

\subsection{Information Flow}
Information flow concerns the development of decentralized algorithms
that route higher-level application information over networks.  Within
information flow, we will study distributed algorithms for
constructing and maintaining routing tables, and gossip-based
information dissemination in highly dynamic networks.  \iffalse We
describe these four technical problems in detail next and include the
review of recently related work along with our technical approaches in
the next section.  \fi

%\smallskip
\BfPara{Minimizing congestion in dynamic networks} The capacity of
links in wireless (ad hoc and sensor) networks often changes due to
fading and multi-path effects.  Further, the mobility of the nodes
themselves leads to changes in the network topology.  A major
challenge in such a dynamic environment is to compute and maintain
bottleneck-free routes for information flow.  There are two main
sources of difficulty: first, the routes must satisfy certain degree
constraints to for keeping routing information within manageable
limits~\cite{xxx}; second, each node only has knowledge of the state
of links to which they are connected, while link capacities can be
dynamically varying.  Given a traffic demand matrix over the set of
nodes, our goal is to devise a distributed algorithm that computes and
maintains {\em congestion minimizing degree-bounded flows}\/ between
all source-destination pairs.  The algorithm must be rapidly
convergent and must utilize minimal control information to avoid
congesting the links that must be conserved for data transport.

\iffalse
since
degree-constraints are necessary for   This motivates the problem of {\em
  congestion minimizing degree-bounded flows}.  We are given a
distributed network $G = (V,E)$.  Each node . Each node also has restrictions on the number of
(outgoing, incoming, both) links it can communicate on. 

We wish to investigate
the problem of finding congestion minimizing flows in such dynamically
varying networks.
\fi

\smallskip
\BfPara{Information dissemination in adversarially dynamic networks}
The dynamics of networks arising in military and critical
infrastructure settings are not only extensive, but can also be
influenced by external adversaries.  A natural model is that of an
adversarial {\em time-evolving network}: in step $t$, the network $G_t
= (V_t, E_t)$ is an arbitrary graph over the set $V_t$ of $n$ nodes,
where the edge sets $E_t$ are determined by an adversary.  Consider a
basic {\em $k$-broadcast problem} in such a setting.  Suppose $k$ of
the $n$ nodes each have a token and would like to disseminate them to
every node in the network.  Further suppose that the nodes are
synchronized and in each step, each node can broadcast the equivalent
of a bounded number of tokens to its neighbors~\cite{kuhn+lo:dynamic}.
What is the minimum number of steps needed to complete the
dissemination?  If the network is completely static and connected,
then a local token-forwarding process on a spanning tree of the
network can accomplish the task in $O(n + k)$ steps, independent of
the structure of the network.  In a dynamic network as in the above
model however, the problem is much more challenging.  A fundamental
open problem is the following: Can the $k$-broadcast problem be solved
on an dynamic, always-connected, $n$-node network be solved in $O(n +
k)$ steps?

\section{Technical Approaches and Recent Related Work}
The optimization problems raised in this proposal pose two major
challenges.  First, how do we deal with uncertainty, whether it is in
the node pairs being connected in the survivable network design
problem, the location of targets in the aggregation tree problem, or
the dynamic network topology in the information flow problems?
Second, how do we ensure that our algorithms can be implemented with
little coordination among the network nodes?

We plan to attack the issue of uncertainty by developing algorithms
for a spectrum of models spanning from {\em stochastic}, where the
unknown online inputs are assumed to be drawn from some known
probability distribution, to {\em adversarial}, where the inputs or
network dynamics are completely under adversarial control.  For the
design of fully-distributed algorithms, we propose to study new {\em
  distributed methods}\/ for constructing network decompositions that
are at the heart of many effective network design algorithms.  We will
also focus on gossip-style algorithms for information flow, which are
inherently decentralized, and have the promise of achieving
near-optimality in highly dynamic environments.  We now elaborate on
the specific technical approaches for the problems listed in
Section~\ref{sec:goals}.

\subsection{Network Design}
\BfPara{Survivable Network Design}  While the survivable network
design problem is known to admit good approximation algorithms in the
offline case, work in the online setting has mainly focused only on
one-connected problems such as Steiner and generalized Steiner
trees~\cite{qw11}. Indeed, no online algorithms were known for higher
connected versions of the problem until our recent
work~\cite{gkr10}. Our algorithms rely on certain tree embeddings and
are inherently centralized.  We propose to study decentralized
algorithms for survivable network design in an online setting where
the underlying graph and costs are static while the connectivity
requirements change online.

\smallskip
\BfPara{Aggregation trees} Consider the problem of building an
aggregation tree for tracking targets.  In a stochastic model of
uncertainty, we may view each target as an object that is detected by
one of the various agents according to a previously determined
probability distribution.  In order to build a backbone network of the
graph for coordination and control, we need to determine a subtree of
the graph containing the root such that given the probability
distribution of target materialization, the subtree maximizes the
expected number of targets that can be covered by the agent nodes in
the tree. If this backbone tree is a spanning tree, then all targets
(each of which materializes at any one of the nodes) will be
covered. At the other end, if no backbone or tree edges can be built a
priori, the root can only detect targets that materialize in its
vicinity. We thus have a trade-off between the allowed size of the
Steiner tree and its expected target coverage that we propose to study
in detail.

We call this problem the \noindent{\bf \em Target Maximizing Rooted
  Steiner Tree} problem.  We are given an undirected graph $G =
(V,E)$, a root node $r$, and a set of targets $T_i, i = 1 \ldots
t$. Each target $T_i$ is a probability distribution $p_i$ over $V$
where $p_i(v)$ is the probability that the target $i$ is at node $v$
(Note that $p_i(v) \geq 0 \ \forall v$ and $\sum_{v \in V} p_i(v) = 1
\ \forall i$). Assume that the distributions $p_i$ and $p_j$ for $i
\neq j$ are independent of each other. Given a target size $S$, our
goal is to find a tree $F$ containing the root of at most $S$ edges
such that the expected size of the targets that materialize in the
node set of the tree is maximized.

This problem can be generalized in various ways. First, the size bound
$S$ can be extended to take into account distances between nodes or
the relative costs of establishing connection between these pairs of
agents. Second, the objective function can be changed to reflect other
notions of reward such as the probability of detecting at least $N$
targets for a given $N$. Third, the independence assumption on target
materialization can be relaxed; this would not make any difference for
the expected target size function but is interesting for other
objectives. Fourth, the tree construction process can be modeled as
the vehicle tour of a control packet in the ad hoc network, leading to
vehicle routing problems that try to minimize expected size to cover
all targets say. Fifth, the notion of stochastic uncertainty can be
replaced by the requirement that the method is robust towards the
worst of many possible scenarios of target materialization. We
investigate one particularly strong notion of robustness for this
problem next.

\smallskip
\noindent {\bf \em Universal approximations for Steiner trees.} We
also plan to study aggregation trees under an adversarial model of
uncertainty.  We recently introduced the framework of {\em universal
  approximations}\/ that provides a robust notion of quality with
respect to {\em any} online sequence of arrivals~\cite{jia+lnrs:universal}.
Universality is a framework for dealing with uncertainty by
guaranteeing a certain quality of goodness for all possible
completions of the partial information set.  \iffalse Universal
variants of optimization problems were defined that were both natural
and well-motivated.\fi We have considered universal versions of three
classical problems: TSP, Steiner Tree and Set Cover.  Subsequent
studies have considered universal variants of other problems
\cite{jia+nrs:gist}. The notion of universality is captured by the
complexity class $\Sigma_2^P$.  Previous work has shown that for any
metric space, there exist Steiner trees that have a universal
approximation ratio of $O(\log^2 n)$~\cite{gupta+hr:oblivious}.  Though these
results reveal new onsights to the structural properties of Steiner
trees and metrics, a major drawback is that they do not apply to
arbitrary graphs.  In recent work, we have made some progress on this
front~\cite{busch+drrs:ust}.  Major problems still remain, including the task of
developing a decentralized method for constructing sparse network
partitions that are required in universal Steiner trees.

In this project, we will explore new paradigms for effective network
design in scenarios where the input may not be known completely in
advance.  Such uncertainty may arise due to limited knowledge about
the future or limited access to global information that may be
distributed among multiple network nodes.  Towards this end, we seek
{\em oblivious algorithms}\/ in which individual network nodes have to
make their decisions with only limited information about the input.
Recent research has highlighted the promise of oblivious algorithms
for some specific network optimization problems.  These include the
seminal result of Racke on oblivious
routing~\cite{azar-cohen,racke-focs}, the work of Goel-Estrin on data
aggregation with unknown aggregation
functions~\cite{goel+e:aggregate}, our recent work introducing the
universal approximations framework~\cite{jia+lnrs:universal}, and the
oblivious network design framework of Gupta et
al.~\cite{gupta+hr:oblivious}.

A major goal of this project is to develop a theory for quantifying
the price of obliviousness in network design.  Towards this end, we
will study formal models for oblivious algorithms, including our
universal approximations framework~\cite{jia+lnrs:universal} and the
oblivious network design framework of~\cite{gupta+hr:oblivious}; we
discuss these models in Section~\ref{sec:universal.models}.  In
Section~\ref{sec:universal.theory}, we discuss the theoretical
foundations of the universal approximations framework.  These include
universal versions of the Steiner tree and covering problems, a new
class of graph decompositions that may have many other applications, a
complexity-theoretic study of universal approximations, and comparison
with other models for optimization with limited information.  Finally,
in Section~\ref{sec:universal.application}, we discuss the application of
oblivious network design to data dissemination in distributed
networks.

\BfPara{Oblivious network design}
In recent work, Gupta et al.\ introduced the following class of
problems in network design~\cite{gupta+hr:oblivious}.  We are given a
network $G = (V,E)$ in which source-sink pairs $\{(s_i,t_i)\}$ arrive,
each pair sending a unit of flow between themselves.  The total cost
of a flow is given by $\sum_{e \in E} \ell(f_e)$, where $f_e$ is the
flow on edge $e$ and $\ell$ is some concave cost function.  The goal
is to select routes for the terminal pairs so as to minimize either
the total cost or the maximum cost incurred.  In a surprising set of
results, Gupta et al.\ show that for both of these objectives,
polylogarithmic approximation ratios are achievable by {\em oblivious                                                    
algorithms}, in which each terminal pair selects a route without any
knowledge of the current flow in the network or of the identity of the
other source-sink pairs, or even of the particular cost function
$\ell(\cdot)$ as long as it is within a certain class.

Formally, an instance of the Universal Steiner Tree (\ust) problem is
a pair $\langle G , r\rangle$ where $G = (V,E)$ is a weighted
undirected graph, and $r$ is a distinguished vertex in $V$ that we
refer to as the {\em root}.  For any spanning tree $T$ of $G$, define
the {\em stretch}\/ of $T$ as $\max_{S \subseteq V}                                                                      
\cost{T_{S\cup r}}/\cost{\OptSt{S\cup r}}$, where $\OptSt{X}$ is an optimal
Steiner tree connecting the vertices in $X \cup r$.  The goal of the
\ust\ problem is to determine a spanning tree with minimum stretch.

Similar to the \ust\ problem, we can define universal versions of
Traveling Salesperson Problem (\utsp), group Steiner tree, group TSP,
and generalized Steiner network problems.  An instance of \utsp\ is a
metric space $(V,d)$.  For any cycle (tour) $\TR$ containing all the
vertices in $V$ and a subset $S$ of $V$, let $\TR_S$ denote the unique
cycle over $S$ in which the ordering of vertices in $S$ is consistent
with their ordering in $C$.  The stretch of $\TR$ is defined as
$\max_{S\subseteq V}                                                                                                     
\cost{\TR_{S}}/\cost{\Optr{S}}$, where $\Optr{S}$ denotes the minimum
cost tour on set $S$. The \utsp\ is to find a tour on $V$ with minimum
stretch.

\Research{
\label{prob:ust}
What are the best stretch achievable for \ust, \utsp, and universal
variants of the group Steiner tree, group TSP, and generalized Steiner
network problems?}

\junk{It is easy to show that a minimum spanning tree may have stretch
$\Omega(n)$ for an $n$-node graph.  Furthermore, even though the
NP-complete minimum Steiner tree problem~\cite{garey+j:np} admits
simple efficient constant-factor approximation
algorithms~\cite[Chapter 3]{vazirani:approximateBook},
~\cite{robins+z:steiner}, one cannot easily extend these
approximations to a universal Steiner tree because any solution needs
to apply to an exponential number of terminal subsets.}

In our ongoing investigation of Problem~\ref{prob:ust}, we have
considered a metric version of the \ust\ problem; that is, the graph
$G$ is the complete graph induced by a metric
space~\cite{jia+lnrs:universal}.  We have developed a polynomial time
algorithm that achieves $O(\log^4n/\log\log n)$ stretch for any $n$
vertex metric space.  Thus, our algorithm finds, for any metric space
over $n$ vertices and a root vertex, a spanning tree $T$ in
polynomial-time such that {\em for every subset $S$ of vertices}\/
containing the root, the sub-tree of $T$ induced by $S$ is {\em within
$O(\log^4n/\log\log n)$ of an optimal Steiner tree for $S$}.

One of the major challenges in constructing a universal Steiner tree
is that any tree will be forced to place some vertices far apart in
tree, even though they may be ``nearby'' according to the underlying
metric distances.  As a result, the resulting spanning tree may
perform poorly on a subset that includes this set and some other
carefully chosen vertices.  To address this challenge, we have
developed a new partitioning scheme, building on the seminal work of
Awerbuch and Peleg on sparse
partitions~\cite{awerbuch+p:partition,peleg:distributeBook}, in which
every neighborhood around any vertex intersects a small number of
partition sets.  More formally, we show that for any integer $\rho$,
any $n$-vertex metric space can be partitioned such that (i) the
diameter of every set in the partition is $O(\rho\log n)$; and (ii)
the ball of radius $\rho$ around any vertex intersects $O(\log n)$
sets in the partition.  By varying the parameter $\rho$, our
partitioning scheme leads to a natural hierarchical decomposition of
the metric space, which forms the key ingredient of our universal
Steiner tree.

For the special case of doubling metrics, our algorithm achieves an
$O(\log n)$ upper bound.  We have also shown a lower bound of
$\Omega(\log n/\log\log n)$ for \ust\ that holds even when all the
vertices are on the plane.  While this settles the best stretch
achievable for Euclidean \ust\ up to a small $O(\log                                                                     
\log n)$ term, there is still a considerable gap between the upper and
lower bounds for the metric case.  Our polylogarithmic upper bounds
also extend to the universal TSP problem and the universal Steiner
forest problems.  In~\cite{gupta+hr:oblivious}, $O(\log^2n)$-stretch
algorithms are given for \utsp\ and a variant of the \ust\ problem in
the oblivious network design framework.

Unfortunately, our result for the metric case does not yield any
useful bounds for the \ust\ problem on general graphs.  Our first line
of attack for this problem is to extend our sparse partitioning
scheme, which we have developed for a metric space, to arbitrary
undirected graphs.  More specifically, we ask the following question:

\Research{
\label{prob:partition}
For any undirected weighted graph $G$ and real number $\rho$, is there
a partition of the vertices of $G$ such that (i) for every set $S$ of
the partition, the subgraph of $G$ induced by $S$ has strong diameter
$O(\rho\log n)$ and (ii) the ball of radius $\rho$ around any node
intersects $O(\log n)$ sets.
}

While an affirmative answer to the above partitioning problem appears
to be neither necessary nor sufficient for solving \ust, we believe
this problem is of great importance with applications in distributed
computing and network design that go beyond the scope of this project.
The seminal results of Awerbuch-Peleg~\cite{awerbuch+p:partition},
Linial-Saks~\cite{linial+s:decomposeJ}, and others (e.g., see the
text~\cite{peleg:distributeBook}) all build graph partitions that have low {\em                                      weak diameter}, as opposed to strong diameter.  In our preliminary
investigation, we have established the claim in the affirmative for
the special case of trees (easy, though not trivial) and outerplanar
graphs (for $\rho = 1$).  The techniques in the recent results
of~\cite{amir+kr:vertex-cut,elkin+est:tree,emek+p:tree} that deal with
some of the strong diameter issues in a different context may be
helpful in this regard.


\subsection{Information Flow}
\BfPara{Congestion-minimizing degree-bounded flows in dynamic
  networks} In earlier work \cite{awerbuch+l:flow}, distributed
algorithms for finding congestion minimizing flows in certain dynamic
networks were presented.  However the routes undertaken by these flows
were unconstrained and expensive to maintain.  We plan to build on our
past work on efficient algorithms for converting optimal flows into
near-optimal degree-bounded
flows~\cite{chen+klrsv:confluent,chen+smr:confluent}; our current
algorithms involve some refinement techniques that require global
knowledge.  We believe that the local balancing approach
of~\cite{awerbuch+l:flow} in conjunction with our flow refinement
techniques will enable us to maintain near-optimal degree-bounded
flows in dynamic networks.  We also plan to consider generalizations
of the problem incorporating link errors and coding techniques.

\iffalse
This problem can be generalized in a multiplicity of ways. Rather than
one commodity one can consider multiple commodities, each with their
own target or destination nodes. Second, one can consider the
availability of end-to-end capacities assuming duplex
connections. Third, one can assume different forms of error that
affect link capacities and the use of techniques such as network
coding. Finally, one can also consider integrating the route discovery
problem with the problem of optimizing the flows.
\fi

\smallskip
\BfPara{Information dissemination in highly dynamic networks} 
We will conduct a comprehensive study of fully-distributed gossip
algorithms for the $k$-broadcast problem in dynamic networks.  In
recent work~\cite{dutta+prs:dynamic}, we studied the class of {\em
  forwarding}\/ algorithms that do not manipulate tokens in any way
other than copying, storing, and forwarding them.  We show that {\em
  any}\/ forwarding algorithm will take $\Omega(nk/\log(n))$ steps to
complete $k$-broadcast, thus resolving an open problem
of~\cite{kuhn+lo:dynamic}.  Given that almost any local greedy
forwarding procedure completes $k$-broadcast in $O(nk)$ steps in any
dynamic network, our lower bound essentially captures the severe
limitations imposed by highly adversarial network dynamics.

A natural and attractive alternative to forwarding algorithms is to
use coding (either end-to-end~\cite{Byers02adigital,Shok06} or
network~\cite{ahlswede+cly:coding}).  Recent
work~\cite{haeupler:gossip,haeupler+k:dynamic} has shown that
information spreading based on network coding can solve $k$-broadcast
in $O(n+k)$ steps, assuming the sizes of the messages are $\Omega(n
\log n)$ bits.  While this message size lower bound is prohibitively
large and impractical (since it scales with the size of the network),
our lower bound~\cite{dutta+prs:dynamic} together with this upper
bound establish, in theory, a fundamental gap between flow-based and
coding-based dissemination procedures.

We propose to study fundamental information spreading problems
on the dynamic network models outlined above, and will quantify our
results in terms of relevant parameters including locality of the
dynamics, conductance/expansion of the evolving network, initial
entropy of the information distribution, and knowledge available to
the dissemination algorithm.

One approach we plan to pursue is to consider a hybrid
forwarding-coding algorithm in which nodes exchange information in the
symmetric difference of what they currently hold, which can be done in
$O(\log n)$ rounds of communication using
fingerprinting~\cite{mitzenmacher-2005-fastmixing}.  We have shown 
that if the entropy of the initial distribution of information
is high, then convergence to full dissemination is rapid.  We will
also consider weaker notions of the adversary (e.g., offline or
oblivious), which model real-world settings where the adversary has
significant control but is not congnizant of all the network actions.
The offline dissemination problem can be reduced to the problem of
packing Steiner trees in directed
graphs~\cite{cheriyan+s:steiner,dutta+prs:dynamic}, and thus has deep
connections with the long-standing open problems of approximating
directed Steiner
trees~\cite{charikar+ccdgg:steiner,zosin+k:steiner,halperin+k:steiner}
and bounding the network coding advantage in multicast over directed
networks~\cite{agarwal+c:coding,sanders+et:flow}.

We expect to quantify our results in terms of relevant parameters
including locality of the dynamics, conductance/expansion of the
evolving network, initial entropy of the information distribution, and
knowledge available to the dissemination algorithm.  We have also
shown that the dissemination problem can be reduced to the problem of
packing Steiner trees in directed graphs, and thus has deep
connections with the long-standing open problems of approximating
directed Steiner trees~\cite{charikar+ccdgg:steiner} and bounding the
network coding advantage in multicast over directed networks.

One approach we plan to pursue is to consider a hybrid
forwarding-coding algorithm in which nodes exchange information in the
symmetric difference of what they currently hold, which can be done in
$O(\log n)$ rounds of communication using
fingerprinting~\cite{mitzenmacher-2005-fastmixing}.  We have shown
that if the entropy of the initial distribution of information is
high, then convergence to full dissemination is rapid.  We will also
consider weaker notions of the adversary (e.g., offline or oblivious),
which model real-world settings where the adversary has significant
control but is not congnizant of all the network actions.  

\iffalse
\section{Recent Related Work}

RELATED WORK: karger-minkoff maybecast tree, gupta-nagarajan-ravi
paper on adaptive TSP (icalp 10), and on stochastic vrp (OR, 2012) and
refs in this for previous papers on TSP with independent demands on
nodes (Bertsimas cycle heuristic). Also the basic expected target max
is like orienteering with size bound on tree, and hence related to
k-MST. Also related to garg-gupta-leonardi-sankowsi (SODA 08) which
relates to Univ TSP.
\fi

\section{Funding Plan}

Our budget requests \$150,000 per year mainly to fund the effort of
the PIs and co-PIs in the form of summer support -- 1 month for the PI
and 0.4 month for each of the co-PIs -- and additional funding for
supporting two graduate students (one student at CMU and one student
at Northeastern).  

%%%%%%%%%%%%%%%%%%%%%%%%%%%%%%%% bibliography %%%%%%%%%%%%%%%%%%%%%%%%%%%%%%

{\small

\begin{thebibliography}{99}

\bibitem{awerbuch+l:flow}
B.~Awerbuch and F.~T. Leighton.
\newblock Improved approximation algorithms for the multi-commodity flow
  problem and local competitive routing in dynamic networks.
\newblock In {\em ACM STOC}, 1994.

\bibitem{bl97}
J. Birge and F. Louveaux. {\em Introduction to Stochastic
Programming}, Springer, Berlin, 1997.

\bibitem{be98}
A. Borodin and R. El-Yaniv. {\em Online Computation and Competitive Analysis}, Cambridge University Press, 1998.

\bibitem{busch+drrs:ust} C.~Busch, C.~Dutta, J.~Radhakrishnan,
  R.~Rajaraman, and S.~Srivathsan.
\newblock Split and Join: Strong Partitions
  and Universal Steiner Trees for Graphs.
\newblock CoRR abs/1111.4766: (2011); under submission.

\bibitem{charikar+ccdgg:steiner}
M.~Charikar, C.~Chekuri, T.~Cheung, Z.~Dai, A.~Goel, and S.~Guha.
\newblock Approximation algorithms for directed steiner problems.
\newblock {\em Journal of Algorithms}, 1998.

\bibitem{chen+klrsv:confluent}
J. Chen, R. D. Kleinberg, L. Lov\'{a}sz, R. Rajaraman, R. Sundaram, A. Vetta.
\newblock (Almost) Tight bounds and existence theorems for single-commodity confluent flows. 
\newblock J. ACM 54(4): (2007)
 
\bibitem{chen+smr:confluent}
J.~Chen, R.~Sundaram, M.~V. Marathe, R.~Rajaraman.
\newblock The Confluent Capacity of the Internet: Congestion vs. Dilation. 
\newblock ICDCS 2006.  {\em Winner of best paper award.}

\bibitem{dgrs05} Kedar Dhamdhere, Vineet Goyal, R. Ravi, Mohit Singh: How to Pay, Come What May: Approximation Algorithms for Demand-Robust Covering Problems. FOCS 2005: 367-378

\bibitem{dutta+prs:dynamic}
C.~Dutta, G.~Pandurangan, R.~Rajaraman, and Z.~Sun.
\newblock Information spreading in dynamic networks, December 2011.
\newblock arXiv:1112.0384; under review.

\bibitem{gupta+hr:oblivious}
A.~Gupta, M.~T. Hajiaghayi, and H.~{R\"{a}cke}.
\newblock Oblivious network design.
\newblock In {\em Proceedings of ACM-SIAM SODA}, pages 970--979, 2006.

\bibitem{haeupler+k:dynamic}
B.~Haeupler and D.~Karger.
\newblock Faster information dissemination in dynamic networks via network
  coding.
\newblock In {\em ACM PODC}, 2011.

\bibitem{jia+lnrs:universal}
L.~Jia, G.~Lin, G.~Noubir, R.~Rajaraman, and R.~Sundaram.
\newblock Universal approximations for TSP, Steiner tree, and Set cover.
\newblock In {\em Proceedings of ACM STOC}, pages 386--39, 2005.

\bibitem{jia+nrs:gist}
L.~Jia, G.~Noubir, R.~Rajaraman, R.~Sundaram.
\newblock
GIST: Group-Independent Spanning Tree for Data Aggregation in Dense Sensor Networks. 
\newblock DCOSS 2006: 282-304.

\bibitem{kuhn+lo:dynamic}
F.~Kuhn, N.~Lynch, and R.~Oshman.
\newblock Distributed computation in dynamic networks.
\newblock In {\em ACM STOC}, pages 513--522, 2010.

\bibitem{gkr10} Anupam Gupta, Ravishankar Krishnaswamy, R. Ravi: Tree Embeddings for Two-Edge-Connected Network Design. SODA 2010: 1521-1538.

\bibitem{qw11} Jiawei Qian, David P. Williamson: An $O(\log
  n)$-Competitive Algorithm for Online Constrained Forest
  Problems. ICALP (1) 2011: 37-48.

\bibitem{rs06}
R. Ravi and A. Sinha:   Hedging uncertainty: Approximation
algorithms for stochastic optimization problems. Math. Program. 108(1):97-114 (2006).
\end{thebibliography}

}

\end{document}
