\section{Project Schedule and Milestones}

The schedule for this project is given in the
Tabel~\ref{table:timeline}.  We expect each topic to span
approximately 4 years of the project.  The total effort is evenly
distributed over the 5 year span of the project.

\begin{table}[h]
\begin{tabular}{||l|c|c|c|c|c|l||}
\hline\hline
{\bf Topic} & {\bf Y1} & {\bf Y2} & {\bf Y3} & {\bf Y4} & {\bf Y5} & {\bf PIs (lead in bold)}\\\hline\hline
Survivable network design & \multicolumn{4}{>{\columncolor[rgb]{0.650000,0.741000,0.858000}}r}{}  & &
%\multicolumn{1}{>{\columncolor[black]{0.8}}c}{} &
%\multicolumn{1}{>{\columncolor[black]{0.8}}c}{} &
%\multicolumn{1}{>{\columncolor[black]{0.8}}c}{} &
%\multicolumn{1}{>{\columncolor[black]{0.8}}c}{} &
{\bf Ravi}, Rajaraman\\\hline
Stochastic aggregation tree problems & & \multicolumn{4}{>{\columncolor[rgb]{0.650000,0.741000,0.858000}}r}{} & {\bf Ravi}, Sundaram\\\hline
Universal aggregation trees  & \multicolumn{4}{>{\columncolor[rgb]{0.650000,0.741000,0.858000}}r}{}  & & {\bf Sundaram}, Rajaraman\\ \hline
Congestion-minimizing degree-bounded flows & & \multicolumn{4}{>{\columncolor[rgb]{0.650000,0.741000,0.858000}}r}{} & {\bf Sundaram}, Ravi\\ \hline
Information dissemination in adversarial networks & \multicolumn{4}{>{\columncolor[rgb]{0.650000,0.741000,0.858000}}r}{} &  & {\bf Rajaraman}, Ravi\\ \hline
\end{tabular}
\caption{Project timeline \label{table:timeline}}
\end{table}

\smallskip
\BfPara{Milestones} We will produce annual reports at the end of each
of the project years.  During the course of the project, we will
document all of the problem formulations and results in papers,
following ONR guidelines for publication.  At a high level, we will
have the following milestones for each of the topics.  For each
milestone, we include in parenthesis its deadline during the 4-year
period devoted to the topic:

\begin{itemize}
\item
{\sl Problem formulation:} Precise formulations for the core problem
and its variants.  (End of 1st year)
\item
{\sl Preliminary bounds:} Basic bounds on the competitive ratio. (End of 1st year)
\item
{\sl Results for online models:} Results for suitable online stochastic and
online adversarial models. (End of 2nd and 3rd years)
\item
{\sl Decentralization:} Quantify the price of decentralization. (End of 4th year)
\item
{\sl Extensions:} Adapt results to model extensions. (End of 5th year)
\end{itemize}

\section{Management Approach}

The whole research team, consisting of the three PIs and all graduate students working on the project will coordinate via bi-weekly meetings arranged online (through a service such as Skype). Research progress in the form of meeting reports, working papers, and any experimental code generated will be shared through a project repository online. A face-to-face annual meeting of the project members will be held every year just before the annual report is due to gather all the information for the report and re-evaluate the project goals and schedule - If appropriate, we will plan this meeting as part of a visit to a professional meeting that we all attend.


\section{Current and Pending Support}

\subsection{Current NSF-EAGR award for R. Ravi}
\begin{enumerate}
\item Title of Proposal and Summary: {\bf EAGER: New Techniques for Graph-TSP}. The traveling salesperson problem (TSP) is a benchmark problem for combinatorial optimization that asks for a shortest tour that visits all the cities in a given network. The graph version where the distances arise from an underlying undirected graph captures a significant portion of the difficulty of designing good algorithms for solving this problem. This proposal will develop a consolidated understanding of the new techniques used in recent developments in the design of improved approximation algorithms for the graph-TSP problem and suggest new ones to move towards optimal performance guarantees.
\item Source and amount of funding: NSF CISE directorate CCF division award number 1143998, for \$99,277.
\item Percentage effort devoted: 2 summer months in 2012
\item Identity of prime Offeror: Carnegie Mellon University, Pittsburgh.
\item Technical Contact: Prof. Ramamoorthi Ravi,  Tepper School of Business at
 Carnegie Mellon University,  5000 Forbes Avenue, Pittsburgh PA 15213-3890.  Tel: +1 412 268 3694;  Fax: +1 412 268 7345;  Email: ravi@cmu.edu.
\item Administrative Contact: Richard Ling, Tepper School of Business at
 Carnegie Mellon University,  5000 Forbes Avenue, Pittsburgh PA 15213-3890. Tel: +1 412 268 5915; Fax: +1 412 268 2810; Email: rjling@andrew.cmu.edu
\item  Period of performance: September 1, 2011 - August 31, 2012.
\item Relation to current proposal: There is no overlap with the currently proposed effort.
\end{enumerate}

\subsection{Pending NSF proposal for R. Ravi}
\begin{enumerate}
\item Title of Proposal and Summary: {\bf AF: Small: Approximation Algorithms for Network Design}. Problems in network design have a two-fold importance in current practice and theory. The key societal advances of the last decade are enabled by the surge of interconnection technologies that have given rise to a variety of such problems. Examples include the design, analysis and operation of supply chain networks, social networks and telecommunication networks. On the other hand, theoretical problems in network design have served as model problems for developing new techniques in the evolution of combinatorial optimization in general and in the design of approximation algorithms in particular. Examples include the primal-dual and iterative methods. This proposal will study several fundamental connectivity problems in network design that have already spurred much algorithmic advance; The focus will be on the traveling salesperson problem and its closely related variants such as the bridge connectivity augmentation problem, two-edge-connected subgraph problem and vertex connectivity network design problems; The goal is to design improved approximation algorithms by advancing current techniques and developing new methods. As a second goal, we propose two new models for network design problems that take into account subadditive demands, and that integrate inventory storage and vehicle routing costs. Thus the two main thrusts of this proposal are to develop new techniques by studying fundamental connectivity problems in network design and new models that will further necessitate such new methods as well as extend and improve existing techniques.
\item Source and amount of funding requested: NSF CISE directorate CCF division proposal number 1218382, for \$494,050.
\item Percentage effort devoted: 2 summer months in 2013, 2014 and 2015.
\item Identity of prime Offeror: Carnegie Mellon University, Pittsburgh.
\item Technical Contact: Prof. Ramamoorthi Ravi,  Tepper School of Business at
 Carnegie Mellon University,  5000 Forbes Avenue, Pittsburgh PA 15213-3890.  Tel: +1 412 268 3694;  Fax: +1 412 268 7345;  Email: ravi@cmu.edu.
\item Administrative Contact: Richard Ling, Tepper School of Business at
 Carnegie Mellon University,  5000 Forbes Avenue, Pittsburgh PA 15213-3890. Tel: +1 412 268 5915; Fax: +1 412 268 2810; Email: rjling@andrew.cmu.edu
\item  Period of performance: September 1, 2012 - August 31, 2015.
\item Relation to current proposal: There is no overlap with the currently proposed effort.
\end{enumerate}
