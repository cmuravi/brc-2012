\section{Project Schedule and Milestones}

The schedule for this project is given in the
Table~\ref{table:timeline}.  We expect each topic to span
approximately 4 years of the project.  The total effort is evenly
distributed over the 5 year span of the project.  The three PIs plan
to work closely on all of the research proposed in this project; the
table lists the lead PI for each of the topics.

\begin{table}[h]
\begin{tabular}{||l|c|c|c|c|c|l||}
\hline\hline
{\bf Topic} & {\bf Y1} & {\bf Y2} & {\bf Y3} & {\bf Y4} & {\bf Y5} & {\bf Lead PI}\\\hline
T1: Survivable network design & \multicolumn{4}{>{\columncolor[rgb]{0.650000,0.741000,0.858000}}r}{}  & &
%\multicolumn{1}{>{\columncolor[black]{0.8}}c}{} &
%\multicolumn{1}{>{\columncolor[black]{0.8}}c}{} &
%\multicolumn{1}{>{\columncolor[black]{0.8}}c}{} &
%\multicolumn{1}{>{\columncolor[black]{0.8}}c}{} &
Ravi\\\hline
T2: Stochastic aggregation tree problems & & \multicolumn{4}{>{\columncolor[rgb]{0.650000,0.741000,0.858000}}r}{} & Ravi\\\hline
T3: Universal aggregation trees  & \multicolumn{4}{>{\columncolor[rgb]{0.650000,0.741000,0.858000}}r}{}  & & Sundaram\\ \hline
T4: Congestion-minimizing degree-bounded flows & & \multicolumn{4}{>{\columncolor[rgb]{0.650000,0.741000,0.858000}}r}{} & Sundaram\\ \hline
T5: Information dissemination in adversarial networks & \multicolumn{4}{>{\columncolor[rgb]{0.650000,0.741000,0.858000}}r}{} &  & Rajaraman\\ \hline
\end{tabular}
\caption{Project timeline \label{table:timeline}}
\end{table}

\smallskip
\BfPara{Milestones} We will produce annual reports at the end of each
of the project years.  During the course of the project, we will
document all of the problem formulations and results in papers,
following ONR guidelines for publication.  At a high level, we will
have the following milestones for each of the topics.  For each
milestone, we include in parenthesis its deadline during the 4-year
period devoted to the topic:

\begin{itemize}
\item
{\sl Problem formulation:} Precise formulations for the core problem
and its variants.  (end of Y1 for T1, T3, and T5; end of Y2 for T4 and T5)
\item
{\sl Preliminary bounds:} Preliminary bounds on the core problem. (end
of Y1 for T1, T3, and T5; end of Y2 for T4 and T5)
\item
{\sl Results for online models:} Results for suitable online stochastic and
online adversarial models. (end of Y2 for T1, T3, and T5; end of Y3 for T4 and T5)
\item
{\sl Decentralization:} Quantify the price of decentralization.  (end
of Y3 for T1, T3, and T5; end of Y4 for T4 and T5)
\item
{\sl Extensions and integration:} Adapt results to model extensions
and integrate the technical results across topics. (end of Y5)
\end{itemize}

\section{Management Approach}

The whole research team, consisting of the three PIs and all graduate
students working on the project will coordinate via bi-weekly meetings
arranged online (through a service such as Skype). Research progress
in the form of meeting reports, working papers, and any experimental
code generated will be shared through a project repository online. A
face-to-face annual meeting of the project members will be held every
year just before the annual report is due, to gather all the
information for the report and re-evaluate the project goals and
schedule.  If appropriate, we will plan this meeting as part of a
visit to a professional meeting that we all attend.

\section{Current and Pending Support}

\subsection{Current awards}

\BfPara{NSF EAGER grant for R. Ravi}
\begin{enumerate}
\item Investigator: R. Ravi
\item Title of Proposal and Summary: {\bf EAGER: New Techniques for Graph-TSP}. The traveling salesperson problem (TSP) is a benchmark problem for combinatorial optimization that asks for a shortest tour that visits all the cities in a given network. The graph version where the distances arise from an underlying undirected graph captures a significant portion of the difficulty of designing good algorithms for solving this problem. This proposal will develop a consolidated understanding of the new techniques used in recent developments in the design of improved approximation algorithms for the graph-TSP problem and suggest new ones to move towards optimal performance guarantees.
\item Source and amount of funding: NSF CISE directorate CCF division award number 1143998, for \$99,277.
\item Percentage effort devoted: 2 summer months in 2012
\item Identity of prime Offeror: Carnegie Mellon University, Pittsburgh.
\item Technical Contact: Prof. Ramamoorthi Ravi,  Tepper School of Business at
 Carnegie Mellon University,  5000 Forbes Avenue, Pittsburgh PA 15213-3890.  Tel: +1 412 268 3694;  Fax: +1 412 268 7345;  Email: ravi@cmu.edu.
\item Administrative Contact: Richard Ling, Tepper School of Business at
 Carnegie Mellon University,  5000 Forbes Avenue, Pittsburgh PA 15213-3890. Tel: +1 412 268 5915; Fax: +1 412 268 2810; Email: rjling@andrew.cmu.edu
\item  Period of performance: September 1, 2011 - August 31, 2012.
\item Relation to current proposal: There is no overlap with the currently proposed effort.
\end{enumerate}

\BfPara{BAE subcontract for R. Rajaraman and R. Sundaram}
\begin{enumerate}
\item Title of Proposal and Summary: {\bf Communications in Extreme RF
  Spectrum Conditions (Commex)}.  This subcontract consists of three
  tasks towards the design of a communication protocol for ad hoc
  networks under extreme adversarial (jamming) conditions: (i) Develop
  a non-cooperative game theory engine, conduct a theoretical
  analysis, and develop the associated algorithms and software; (ii)
  Develop algorithms and techniques that provide optimized protocol
  mechanism hopping at the physical and media access control layers;
  (iii) Perform theoretical analysis and trade studies to characterize
  the performance of jammer jujitsu interference deception algorithms
  against a range of jammer/interference models.
\item Source and amount of funding: BAE Systems subcontract on DARPA
  grant, for \$650K.
\item Percentage effort devoted: 1 summer month in 2012, and 1 course buyout in 2012, for each PI
\item Identity of prime Offeror: Northeastern University
\item Technical Contact: Prof. Ravi Sundaram,  202 WVH, CCIS, Northeastern University
Boston MA 02115.  Tel: +1 617 373 5876;  Fax: +1 617 373 5121;  Email: koods@ccs.neu.edu
\item Administrative Contact: Deborah Grupp-Patrutz, Director,
  Research Administration and Finance, Northeastern University, 360
  Huntington Avenue, 960 RP, Boston, MA 02115. Phone: (617) 373-5600,
  Fax: (617) 373-4595 Email: oraf@neu.edu
\item  Period of performance: July 1, 2011 - February 28, 2013.
\item Relation to current proposal: There is no overlap with the currently proposed effort.
\end{enumerate}

\subsection{Pending proposals (under review)}

\BfPara{NSF AF proposal for R. Ravi}
\begin{enumerate}
\item Title of Proposal and Summary: {\bf AF: Small: Approximation Algorithms for Network Design}. Problems in network design have a two-fold importance in current practice and theory. The key societal advances of the last decade are enabled by the surge of interconnection technologies that have given rise to a variety of such problems. Examples include the design, analysis and operation of supply chain networks, social networks and telecommunication networks. On the other hand, theoretical problems in network design have served as model problems for developing new techniques in the evolution of combinatorial optimization in general and in the design of approximation algorithms in particular. Examples include the primal-dual and iterative methods. This proposal will study several fundamental connectivity problems in network design that have already spurred much algorithmic advance; The focus will be on the traveling salesperson problem and its closely related variants such as the bridge connectivity augmentation problem, two-edge-connected subgraph problem and vertex connectivity network design problems; The goal is to design improved approximation algorithms by advancing current techniques and developing new methods. As a second goal, we propose two new models for network design problems that take into account subadditive demands, and that integrate inventory storage and vehicle routing costs. Thus the two main thrusts of this proposal are to develop new techniques by studying fundamental connectivity problems in network design and new models that will further necessitate such new methods as well as extend and improve existing techniques.
\item Source and amount of funding requested: NSF CISE directorate CCF division proposal number 1218382, for \$494,050.
\item Percentage effort devoted: 2 summer months in 2013, 2014 and 2015.
\item Identity of prime Offeror: Carnegie Mellon University, Pittsburgh.
\item Technical Contact: Prof. Ramamoorthi Ravi,  Tepper School of Business at
 Carnegie Mellon University,  5000 Forbes Avenue, Pittsburgh PA 15213-3890.  Tel: +1 412 268 3694;  Fax: +1 412 268 7345;  Email: ravi@cmu.edu.
\item Administrative Contact: Richard Ling, Tepper School of Business at
 Carnegie Mellon University,  5000 Forbes Avenue, Pittsburgh PA 15213-3890. Tel: +1 412 268 5915; Fax: +1 412 268 2810; Email: rjling@andrew.cmu.edu
\item  Period of performance: September 1, 2012 - August 31, 2015.
\item Relation to current proposal: There is no overlap with the currently proposed effort.
\end{enumerate}
\newpage
\BfPara{NSF ICES proposal for R. Rajaraman and R. Sundaram}
\begin{enumerate}
\item Title of Proposal and Summary: {\bf
  NSF:ICES:CollaborativeResearch: The Role of Space, Time, and
  Information in Controlling Epidemics.} The control of epidemics,
  broadly defined to range from human diseases such as influenza and
  smallpox to malware in communication networks, relies crucially on
  interventions such as vaccinations and anti-virals (in human
  diseases) or software patches (for malware). These interventions are
  almost always voluntary directives from public agencies; however,
  people do not always adhere to such recommendations, and make
  individual decisions based on their specific “self interest”.
  Additionally, people alter their contacts dynamically, and these
  behavioral changes have a huge impact on the dynamics and the
  effectiveness of these interventions, so that “good” intervention
  strategies might, in fact, be ineffective, depending upon the
  individual response.  The goal of this proposal is to study the
  foundations of policy design for controlling epidemics, using a
  broad class of epidemic games on complex networks involving
  uncertainty in network information, temporal evolution and learning.
  We propose models that capture the complexity of static and temporal
  interactions and patterns of information exchange, including the
  possibility of failed interventions and the potential for moral
  hazard.  We also consider specific policies posed by public agencies
  and network security providers for controlling the spread of
  epidemics and malware, and study their efficacy and resource
  constrained mechanisms to implement them in this framework.
\item Source and amount of funding: NSF CISE directorate CCF division
  proposal number 1216038, for \$275,000.
\item Percentage effort devoted: 0.25 summer months for each PI in 2013, 2014, and 2015.
\item Identity of prime Offeror: Northeastern University
\item Technical Contact: Prof. Rajmohan Rajaraman,  202 WVH, CCIS, Northeastern University
Boston MA 02115.  Tel: +1 617 373 2075;  Fax: +1 617 373 5121;  Email: rraj@ccs.neu.edu
\item Administrative Contact: Deborah Grupp-Patrutz, Director,
  Research Administration and Finance, Northeastern University, 360
  Huntington Avenue, 960 RP, Boston, MA 02115. Phone: (617) 373-5600,
  Fax: (617) 373-4595 Email: oraf@neu.edu
\item  Period of performance: September 1, 2012 - August 31, 2015.
\item Relation to current proposal: There is no overlap with the currently proposed effort.
\end{enumerate}

\BfPara{NSF NeTS proposal for R. Rajaraman}
\begin{enumerate}
\item Title of Proposal and Summary: {\bf NeTS: Small: Scaling Wireless
  and Data Access – the Power of Community Networks}.  PI Guevara
  Noubir.  Wireless communication systems have been a key enabler of
  the smart phones revolution. However, such systems are under a
  continuously increasing demand for bandwidth and are reaching their
  limit forcing the operators to deploy, operate, and maintain base
  stations in the users homes (femto-BS) without added incentives to
  the users. At the same time, Wireless Access Points (APs) already
  have an extremely high density in urban areas, operate on unlicensed
  bands (greater than 200Mhz of ISM bands), provide high data rates,
  have high availability, are closer to the users than content
  delivery networks (CDN), consume little energy, and can easily be
  extended to provide storage and caching services. They therefore
  have several characteristics that make them a good candidate for a
  community infrastructure that provides truly ubiquitous wireless and
  data access. However, several fundamental and systems challenges
  have to be addressed to make such networks a reality.  The goal of
  this research is to develop enabling mechanisms, protocols, and
  demonstrators for community networks that provide ubiquitous
  wireless and data access.
\item Source and amount of funding: NSF CISE directorate CNS division
  proposal number 1218146, amount \$499,954.00.
\item Percentage effort devoted: 0.25 summer months in 2013, 2014, and 2015.
\item Identity of prime Offeror: Northeastern University
\item Technical Contact: Prof. Guevara Noubir,  202 WVH, CCIS, Northeastern University
Boston MA 02115.  Tel: +1 617 373 5205;  Fax: +1 617 373 5121;  Email: noubir@ccs.neu.edu
\item Administrative Contact: Deborah Grupp-Patrutz, Director,
  Research Administration and Finance, Northeastern University, 360
  Huntington Avenue, 960 RP, Boston, MA 02115. Phone: (617) 373-5600,
  Fax: (617) 373-4595 Email: oraf@neu.edu
\item  Period of performance: September 1, 2012 - August 31, 2015.
\item Relation to current proposal: There is no overlap with the currently proposed effort.
\end{enumerate}

\BfPara{NSF CSR proposal for R. Rajaraman}
\begin{enumerate}
\item Title of Proposal and Summary: {\bf CSR: Small: Collaborative
  Research: Pursuing High Performance on Clouds and Other Dynamically
  Heterogeneous Computing Platforms}. PI Arnold Rosenberg.  The
  proposed research will develop a transformative computing paradigm
  that will enable high performance computing on modern platforms such
  as (computing) clouds and many genres of (computing) grids. Clouds
  and grids promise to make high-performance computing platforms
  accessible to the public; but realizing this promise requires one to
  cope with the platforms’ dynamic heterogeneity, i.e., the fact that
  their constituent computers’ relative powers and speeds can change
  at unpredicable times and in unpredictable ways. (For instance, a
  computer in a cloud may slow down because it is assigned more work
  from another user.) The new paradigm replaces traditional
  schedulers’ attempts to accommodate the particulars of a computing
  platform—a goal that dynamic heterogeneity confutes—by orchestrating
  a complex computation in a way that honors the relevant details of
  the computation’s inherent structure. Specifically, the paradigm
  aims to enhance the average rate at which the computation’s
  constituent tasks are rendered eligible for execution, by executing
  tasks in an appropriate order. (Because of this goal, the paradigm’s
  schedules are termed area-oriented for reasons detailed in the
  project description.) This strategy aims to: (a) increase
  opportunities for executing independent tasks in parallel and (b)
  minimize the chance of the computation’s stalling pending completion
  of already allocated tasks.
\item Source and amount of funding: NSF CISE directorate CNS division
  proposal number 1217981, amount \$315,513.
\item Percentage effort devoted: 0.25 summer months in 2013, 2014, and 2015.
\item Identity of prime Offeror: Northeastern University
\item Technical Contact: Prof. Arnold Rosenberg,  202 WVH, CCIS, Northeastern University
Boston MA 02115. Fax: +1 617 373 5121;  Email: rsnbrg@ccs.neu.edu
\item Administrative Contact: Deborah Grupp-Patrutz, Director,
  Research Administration and Finance, Northeastern University, 360
  Huntington Avenue, 960 RP, Boston, MA 02115. Phone: (617) 373-5600,
  Fax: (617) 373-4595 Email: oraf@neu.edu
\item  Period of performance: September 1, 2012 - August 31, 2015.
\item Relation to current proposal: There is no overlap with the currently proposed effort.
\end{enumerate}
