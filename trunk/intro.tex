\section{Introduction}

As communication devices and sensor infrastructures scale at a rapid
rate, it has now become possible to deploy large ad-hoc networks that
communicate in a limited fashion without a centralized control, and
carry out useful tasks (such as surveillance and threat detection)
even with limited coordination and control. This naturally leads us to
the consideration of {\em distributed} algorithms that use little or
no central control.

The lack of control is further exacerbated by the lack of complete
information about the environment in which the agents are located,
adding an extra layer of uncertainty to the problem. The paucity of
information can be modeled either by providing a probabilistic model
of the information which leads to {\em stochastic optimization}
models~\cite{bl97,rs06}, or by an adversarial model of information
revelation that leads to competitive analysis in the sense of {\em online
optimization} models~\cite{be98}. When the scenarios of uncertainty are neither quantifiable using randomness, nor adversarial, the framework of {\em robust optimization} models~\cite{dgrs05} is also useful.

In this proposal, we develop robust models for studying the
fundamental tasks of establishing and maintaining connectivity and
control in complex networked systems, which capture (i) limited
communication between agents (necessitating distributed algorithms),
(ii) online revelation of information over time (including stochastic,
robust and competitive frameworks) and (iii) permit the development of
rigorous approximation algorithms with a polynomial running time that
provide provably good solutions with performance guarantees. These
three aspects of our model directly address the three main
characteristics in this BRC topic.

\subsection{Summary of technical goals}
%\section{Models}
\label{sec:goals}

The main goal of this project is to develop a comprehensive theory of
{\bf \em distributed online approximation}\/ algorithms for hard network
optimization problems.  
\iffalse (Repeats what is in the prev para)
This proposal encompasses several challenging
aspects of mission-critical networked systems: (a) there is
considerable uncertainty in the inputs and the network environment
under which the systems operate; (b) the inputs as well as the
underlying network may, in fact, be under adversarial control; (c) the
algorithms running these networked systems need to be
fully-distributed.
\fi
At a high level, the problems we plan to study in this project can be
divided into two categories: {\bf \em network design}\/ and {\bf \em
information flow}.  

\BfPara{Network Design} Network design concerns the construction of
overlay network structures that form the foundation for aggregation,
point-to-point routing, broadcast, multicast and other critical
network functions.  The deployment of mission-critical military
systems requires the ability to construct and maintain such
large-scale network structures that will enable secure and reliable
communication and operation in a highly dynamic and distributed
envivornment.  Research in network design has been at the forefront of
major advances in approximation algorithms.  We believe that
theoretical foundations of distributed online network design are
essential to achieve major advances in this area.  Within network
design, we will study the general problem of {\em survivable network
  design}, with a focus on the important special case of {\em
  aggregation trees}.

\begin{itemize}
\item
{\sl Survivable network design:} In the {\em survivable network design
  problem}, we are given a graph $G = (V,E)$ with edge-costs, and
edge-connectivity requirements $r_{ij} \in Z_{\ge 0}$ for every pair
of vertices $i, j \in V$, and need to find an (approximately)
minimum-cost network that provides the required connectivity.  This is
one of the most fundamental problems in network design that
generalizes several graph-theoretic optimization problems including
shortest paths, spanning and Steiner trees.  The edge-connectivity
requirements capture the need for increased reliability in
mission-critical systems; furthermore, the general statement of the
problem allows one to develop algorithmic paradigms that may have
broad applicability.

\item
{\sl Aggregation trees:} A special case of survivable network design
is the fundamental problem of constructing a tree that aggregates
information from important agents in a distributed network to a
central point (HQ). The agents are modeled as nodes in a graph, whose
edges capture the communication links between the agents.  Suppose the
primary task of the network is to detect and monitor specific targets
whose location the agents are unaware of a priori.  As the agents
explore their local terrain in detail they may become aware of the
presence of valuable targets. This converts such a node into a {\em
  terminal} node that must now communicate with the HQ node henceforth
called the {\em root}.  This problem is a variant of the classical
minimum Steiner tree problem.  Though Steiner tree and their variants
have been extensively studied, there are no distributed algorithms for
computing near-optimal Steiner trees in dynamic or uncertain
environments.  We propose new distributed methods for constructing
near-optimal aggregation trees in stochastic and adversarial
scenarios.
\end{itemize}

\BfPara{Information Flow}
Information flow concerns the development of decentralized algorithms
that route higher-level application information over networks.  Within
information flow, we will study distributed algorithms for
constructing and maintaining routing tables, and gossip-based
information dissemination in highly dynamic networks.  

\begin{itemize}
\item
{\sl Minimizing congestion in dynamic networks:} The capacity of links
in wireless (ad hoc and sensor) networks often changes due to fading
and multi-path effects.  Further, the mobility of the nodes themselves
leads to changes in the network topology.  A major challenge in such a
dynamic environment is to compute and maintain bottleneck-free routes
for information flow.  There are two main sources of difficulty:
first, the routes must satisfy certain degree constraints to for
keeping routing information within manageable limits~\cite{xxx};
second, each node only has knowledge of the state of links to which
they are connected, while link capacities can be dynamically varying.
Given a traffic demand matrix over the set of nodes, our goal is to
devise a distributed algorithm that computes and maintains {\em
  congestion minimizing degree-bounded flows}\/ between all
source-destination pairs.  The algorithm must be rapidly convergent
and must utilize minimal control information to avoid congesting the
links that must be conserved for data transport.

\item
{\sl Information dissemination in adversarially dynamic networks} The
dynamics of networks arising in military and critical infrastructure
settings are not only extensive, but can also be influenced by
external adversaries.  A natural model is that of an adversarial {\em
  time-evolving network}: in step $t$, the network $G_t = (V_t, E_t)$
is an arbitrary graph over the set $V_t$ of $n$ nodes, where the edge
sets $E_t$ are determined by an adversary.  We propose to develop
purely local lightweight algorithms for fundamental information
dissemination tasks in highly dynamic networks.  Our starting point
will be the basic {\em $k$-broadcast problem} in $k$ of the $n$ nodes
each have a message and would like to disseminate them to every node
in the network.  A fundamental open problem is the following: Can the
$k$-broadcast problem be solved on an dynamic, always-connected,
$n$-node network be solved in $O(n + k)$ steps?
\end{itemize}

\subsection{Relevance to the Navy}

\subsection{PI Information}
Prof. R. Ravi works in the intersection of Operations Research and
Computer Science and has pioneered work in various network
optimization problems such as the Buy-at-Bulk Network Design problem
with economies of scale, the Group Steiner Tree problem that combines
the Set Covering and Steiner Tree problems, and in designing provably
near-optimal approximation algorithms with performance guarantees for
problems in stochastic and robust combinatorial optimization. He has
also advanced the development of tools for the design of approximation
algorithms for network optimization such as the primal-dual method,
dependent randomized rounding, Lagrangean relaxations and iterative
methods.  

Co-PI Rajaraman's research expertise covers distributed computing
theory, approximation algorithms, network optimization, and
algorithmic game theory.  His work on distributed hash tables is
widely-cited and has been implemented in several peer-to-peer network
systems.  Co-PI Sundaram's research expertise covers networks,
algorithms, complexity theory and combinatorics.  Before joining
academics, he was Director of Engineering at Akamai Technologies,
where he established the mapping group for the world's leading content
delivery network, which is responsible for directing browser requests
(over 10 billion a day) to the optimal Akamai server.

The PIs have a strong history of collaboration.  PI Ravi and co-PI
Sundaram were co-authors on a widely cited foundational paper on
bicriteria approximations in network design.  Co-PIs Rajaraman and
Sundaram were co-authors on a ICDCS 2006 paper on Internet capacity
that won the best paper award.  With deep synergistic expertise in
optimization theory, online and approximation algorithms, and
distributed computing, the team is uniquely positioned to address the
basic research challenges of this call.

