\subsection{Network Design}
Our research in network design will concern distributed online
algorithms for the general survivable network design problem under
certain stochastic and adversarial scenarios (Section~\ref{sec:sndp}),
and explore more robust solutions for several variants of aggregation
tree problems that often arise in network applications
(Sections~\ref{sec:aggregation} and~\ref{sec:ust}).

\subsubsection{Survivable Network Design}
\label{sec:sndp}
Recall that in the {\em survivable network design problem (SNDP)}, we
are given a graph $G = (V,E)$ with edge-costs, and edge-connectivity
requirements $r_{ij} \in Z_{\ge 0}$ for every pair of vertices $i, j
\in V$, and need to find an (approximately) minimum-cost network that
provides the required connectivity.  Since SNDP is NP-hard (it
contains the Steiner tree problem as a special case), it has been
widely studied from the viewpoint of approximation algorithms (See
e.g.,~\cite{GK11} for a survey of results). These problems were one of
the earliest applications of the primal-dual method in this area which
led, over a sequence of papers, to the development of an $O(\log
r_{\max})$-approximation algorithm~\cite{GGPSTW94}.  Subsequently, one
of the first uses of iterative rounding in approximation algorithms
led to a $2$-approximation for this problem (and for the general
problem of network design with weakly-supermodular
functions)~\cite{Jain01}.

In~\cite{gkr10}, we studied these problems in the
\emph{online} setting: we are given a graph with edge-costs, and an
upper bound $r_{\max}$ on the connectivity demand. A sequence of vertex pairs $\{i,j\} \in V
\times V$ is presented to us over time, each with some edge-connectivity
demand $r_{ij}$---at this point we may need to buy some edges to ensure
that all the edges bought by the algorithm provide
an edge-connectivity of  $r_{ij}$ between vertices $i$ and $j$.  The goal is to remain
competitive with the optimal offline solution of the current demand set.
Work in this online
setting has mainly focused only on one-connected problems such as
Steiner and generalized Steiner trees~\cite{qw11}.
No online algorithms were previously known
for this problem even for the online rooted $2$-connectivity problem (i.e., for
the case where all the vertex pairs share a root vertex $r$ and the
connectivity requirement is~$2$ for all pairs) while we noted
a lower bound of $\Omega(\min\{|D|, \log n\})$ on the competitive ratio for this special case, where $D$ is the set of terminal pairs given to the
algorithm. This is in contrast to the case of online $1$-connectivity (i.e., online Steiner forest) where the best online algorithm is $\Theta(\log |D|)$-competitive \cite{BCsteiner97}.
Our main result for the edge-connected survivable network design problem is an $O(r_{\max} \log^3 n)$-competitive randomized online algorithm against oblivious adversaries.

Our algorithms use the standard embeddings of graphs into random
subtrees (i.e., into \emph{singly connected} subgraphs) as an
intermediate step to get algorithms for higher connectivity. As a
consequence of using these random embeddings, our algorithms are
competitive only against oblivious adversaries. A natural extension
that is still open is whether our methods can be extended to work
against adaptive adversaries.  Furthermore, our algorithms relying on
tree embeddings and inherently centralized, and it is unclear how to
extend even the one-connected online algorithms for Steiner trees to
the distributed setting.

\Research{
\label{prob:sndp}
Devise distributed online algorithms for the Survivable Network Design
Problem with good competitive ratios against adaptive adversaries.}

We propose to study decentralized algorithms for survivable network
design in an online setting where the underlying graph and costs are
static while the connectivity requirements change online.  To design
decentralized algorithms, we will consider two sets of techniques, one
used for global structures such as minimum spanning
trees~\cite{gallager+hs:mst,garay+kp:mst}, and the other for local
structures such as minimum dominating sets~\cite{jia+rs:dominateFull}.
We believe that a combination of these techniques will be helpful in
tackling several special cases of SNDP where the connectivity
requirements satisfy some locality constraints.  We also plan to
quantify the price of decentralization of SNDP, which will help
determine what network design problems can be effectively solved using
decentralizated algorithms, and which others may even be infeasible to
solve in a decentralized manner.

\junk{ %%% The following seems to be a repeat of the above
Finally, our algorithms are all centralized, and it is unclear how to extend even the one-connected online algorithms for Steiner trees to the distributed setting.
}

Another direction to extend the SNDP is the
stochastic case when the instance is drawn according to a probability
distribution. In ~\cite{gkr10}, we considered the case when we have a product
distribution: for each pair $i,j$ of vertices we are given a probability $p_{ij}$,
and are guaranteed that \emph{tomorrow} each pair will flip their coins
independently, and if the coin turns up heads, they would demand
$k$-connectivity. We can buy some edges today at cost $c(\cdot)$, but if
we wait for the actual set $D$, the edges will cost $\lambda c(\cdot)$
tomorrow, for a pre-specified inflation parameter $\lambda \geq 1$; the goal
is to minimize the sum of the cost of edges bought today and the
expected cost of augmentation edges bought tomorrow (at the inflated price).
In that work, we assumed for simplicity that all pairs have the the same
connectivity requirement of $k$. A simple generalization would allow each arriving pair to specify even the connectivity to be any value in $\{0,1\ldots,k\}$ according to a pre-defined probability distribution -- we'll now have values $p_{ij}(\kappa)$ for $\kappa \in \{0,1\ldots,k\}$, summing to one. We propose to extend our previous work to this more interesting setting that allows for more general stochastic specifications of connectivity requirements.

\subsubsection{Aggregation trees}
\label{sec:aggregation}
Consider the problem of building an aggregation tree for tracking
targets.  In a stochastic model of uncertainty, we may view each
target as an object that is detected by one of the various agents
according to a previously determined probability distribution.  In
order to build a backbone network of the graph for coordination and
control, we need to determine a subtree of the graph containing the
root such that given the probability distribution of target
materialization, the subtree maximizes the expected number of targets
that can be covered by the agent nodes in the tree. If this backbone
tree is a spanning tree, then all targets (each of which materializes
at any one of the nodes) will be covered. At the other end, if no
backbone or tree edges can be built a priori, the root can only detect
targets that materialize in its vicinity. We thus have a trade-off
between the allowed size of the Steiner tree and its expected target
coverage and this is what we propose to study in detail.

We call this problem the \noindent{\bf \em Target Maximizing Rooted
Steiner Tree} problem.  We are given an undirected graph $G =
(V,E)$, a root node $r$, and a set of targets $T_i, i = 1 \ldots
t$. Each target $T_i$ is a probability distribution $p_i$ over $V$
where $p_i(v)$ is the probability that the target $i$ is at node $v$
(Note that $p_i(v) \geq 0 \ \forall v$ and $\sum_{v \in V} p_i(v) = 1
\ \forall i$). Assume that the distributions $p_i$ and $p_j$ for $i
\neq j$ are independent of each other. Given a budget $S$ on the size, our
goal is to find a tree $F$ containing the root of at most $S$ edges
such that the expected size of the targets that materialize in the
node set of the tree is maximized.

This problem can be generalized in various ways. First, the size bound
$S$ can be extended to take into account distances between nodes or
the relative costs of establishing connection between these pairs of
agents. Second, the objective function can be changed to reflect other
notions of reward such as the probability of detecting at least $N$
targets for a given $N$. Third, the independence assumption on target
materialization can be relaxed; this would not make any difference for
the expected target size function but is interesting for other
objectives. Fourth, the tree construction process can be modeled as
the vehicle tour of a control packet in the ad hoc network, leading to
vehicle routing problems that try to minimize expected size to cover
all targets say. Fifth, the notion of stochastic uncertainty can be
replaced by the requirement that the method is robust towards the
worst of many possible scenarios of target materialization. We
investigate one particularly strong notion of robustness for this
problem in Section~\ref{sec:ust}.

\Research{
\label{prob:tmrst}
Devise offline approximation algorithms for the Target Maximizing
Rooted Steiner Tree problem with good constant performance
ratio. Extend the techniques to the various generalizations and to the
robust and distributed settings.}

As stated, the problem is closely related to the $k$-MST problem that
PIs Ravi and Sundaram introduced in an early paper~\cite{RSMRR94}, and
for which we eventually designed the first constant-factor
approximation algorithm~\cite{BRV99}. In this problem we are given an
undirected edge-weighted graph with a root and an integer $k$, and the
goal is to find a minimum-weight spanning tree connecting the root
with at least $k$ other nodes. Our target maximizing version is close
to the complementary version of the problem where we are given a bound
on the cost of the tree and the goal is to maximize the number of
nodes covered. This is also related to the {\em Orienteering Problem}
where the object collecting the nodes is an orienteering path rather
than a tree, and for which constant factor approximations are
known~\cite{BCKLMM03}. We have recently also devised approximation
algorithms for stochastic versions of the orienteering
problem~\cite{GKNR12} but the stochasticity is related to delays at
the nodes to fetch the rewards rather than the location of targets on
the visited nodes. These are all interesting approaches that we
propose to investigate to solve the target maximizing problems we
propose. 

Two other challenging directions that we will pursue are to extend
these centralized methods to work in a distributed setting, and to
bound the price of decentralization for the general problem.  For the
former, we expect that it will be important to have the nodes exchange
appropriate information about the target probability distributions to
one another; in this regard, our proposed local methods for
disseminating information (Section~\ref{sec:spreading}) may be
helpful.

\iffalse
RELATED WORK: karger-minkoff maybecast tree, gupta-nagarajan-ravi
paper on adaptive TSP (icalp 10), and on stochastic vrp (OR, 2012) and
refs in this for previous papers on TSP with independent demands on
nodes (Bertsimas cycle heuristic). Also the basic expected target max
is like orienteering with size bound on tree, and hence related to
k-MST. Also related to garg-gupta-leonardi-sankowsi (SODA 08) which
relates to Univ TSP.
\fi

\subsubsection{Universal approximations for Steiner tree and related problems}
\label{sec:ust}
We also plan to study aggregation trees under an adversarial model of
uncertainty.  We have introduced the framework of {\em universal
  approximations}\/ that provides a robust notion of quality with
respect to {\em any} online sequence of
arrivals~\cite{jia+lnrs:universal}.  Universality is a framework for
dealing with uncertainty by guaranteeing a certain quality of goodness
for all possible completions of the partial information set.
Formally, an instance of the Universal Steiner Tree (\ust) problem is
a pair $\langle G , r\rangle$ where $G = (V,E)$ is a weighted
undirected graph, and $r$ is a distinguished vertex in $V$ that we
refer to as the {\em root}.  For any spanning tree $T$ of $G$, define
the {\em stretch}\/ of $T$ as $\max_{S \subseteq V}\cost{T_{S\cup
    r}}/\cost{\OptSt{S\cup r}}$, where $\OptSt{X}$ is an optimal
Steiner tree connecting the vertices in $X$.  The goal of the
\ust\ problem is to determine a spanning tree with minimum stretch.

Similar to the \ust\ problem, we can define universal versions of
Traveling Salesperson Problem (\utsp), group Steiner tree, group TSP,
and generalized Steiner network problems.  An instance of \utsp\ is a
metric space $(V,d)$.  For any cycle (tour) $\TR$ containing all the
vertices in $V$ and a subset $S$ of $V$, let $\TR_S$ denote the unique
cycle over $S$ in which the ordering of vertices in $S$ is consistent
with their ordering in $C$.  The stretch of $\TR$ is defined as
$\max_{S\subseteq V} \cost{\TR_{S}}/\cost{\Optr{S}}$, where $\Optr{S}$
denotes the minimum cost tour on set $S$. The \utsp\ is to find a tour
on $V$ with minimum stretch.

\Research{
\label{prob:ust}
What are the best stretch achievable for \ust, \utsp, and universal
variants of the group Steiner tree, and special cases of SNDP?}

The notion of universality is captured by the complexity class
$\Sigma_2^P$, and we conjecture that determining the best universal
algorithms for the above problems is $\Sigma^P_2$-hard.  In previous
work, however, we have been able to achieve near-optimal universal
solutions in some cases.  For any metric space, there exist Steiner
trees that have universal poylogarithmic approximation ratios: our
prior results~\cite{jia+lnrs:universal}, and the improved results of
$O(\log^2 n)$~\cite{gupta+hr:oblivious} that introduce the notion of
oblivious network design.  Though these results reveal new insights to
the structural properties of Steiner trees and metrics, a major
drawback is that {\em they do not apply to arbitrary graphs}.

In very recent work, we have made some progress on this
front~\cite{busch+drrs:ust}, and have identified a promising approach
to attack the problem.  One of the major challenges in constructing a
universal Steiner tree is that any tree will be forced to place some
vertices far apart (in the tree), even though they may be ``nearby''
according to the underlying graph distances.  As a result, the
resulting spanning tree may perform poorly on a subset that includes
this set and some other carefully chosen vertices.  To address this
challenge, we introduce the notion of a {\em hierarchy of graph
  partitions}, each of which guarantees small strong cluster diameter
and bounded local neighbourhood intersections.  We have shown that the
such a suitable hierarchy of graph partitions is essentially both
sufficient and necessary for constructing low-stretch universal
Steiner trees.  For metric spaces -- i.e., weighted complete graphs
satisfying triangle inequality -- such a hierarchy can be constructed
using the seminal work of Awerbuch and Peleg on sparse
partitions~\cite{awerbuch+p:partition,peleg:distributeBook}.

It is a major open problem, however, to build such hierarchies for
arbitrary graphs.  We have made preliminary progress by presenting
partition hierarchies for general graphs that yield a $2^{O(\sqrt{\log
    n})}$-stretch \ust\ for general graphs, and partition hierarchies
for minor-free graphs that yield a polylogarithmic-stretch \ust\ for
minor-free graphs.  Furthermore, all of the solutions proposed thus
far are centralized.  We propose to develop distributed algorithms for
constructing sparse partition hierarchies.  In an earlier work on
aggregation trees for sensor network applications, we showed that this
is achievable for the highly specialized case of grid
graphs~\cite{jia+nrs:gist}, thus ensuring a small price of
decentralization for the universal Steiner tree problem on grid-like
graphs.  For general graphs, we plan to build on these ideas and the
techniques developed in~\cite{peleg:distributeBook} for dstributed construction
of related network decompositions.

