\section{Technical Approach and Justification}

As communication devices and sensor infrastructures scale at a rapid
rate, it has now become possible to deploy large ad-hoc networks that
communicate in a limited fashion without a centralized control, and
carry out useful tasks (such as surveillance and threat detection)
even with limited coordination and control. This naturally leads us to
the consideration of {\em distributed} algorithms that use little or
no central control.

The lack of control is further exacerbated by the lack of complete
information about the environment in which the agents are located,
adding an extra layer of uncertainty to the problem. The paucity of
information can be modeled either by providing a probabilistic model
of the information which leads to {\em stochastic optimization}
models~\cite{bl97,rs06}, or by an adversarial model of information
revelation that leads to competitive analysis in the sense of {\em online
optimization} models~\cite{be98}. When the scenarios of uncertainty are neither quantifiable using randomness, nor adversarial, the framework of {\em robust optimization} models~\cite{dgrs05} is also useful.

In this proposal, we develop robust models for studying the
fundamental tasks of establishing and maintaining connectivity and
control in complex networked systems, which capture (i) limited
communication between agents (necessitating distributed algorithms),
(ii) online revelation of information over time (including stochastic,
robust and competitive frameworks) and (iii) permit the development of
rigorous approximation algorithms with a polynomial running time that
provide provably good solutions with performance guarantees. These
three aspects of our model directly address the three main
characteristics in this BRC topic.

\subsection{Technical Goals}
%\section{Models}
\label{sec:goals}

The main goal of this project is to develop a comprehensive theory of
{\bf \em distributed online approximation}\/ algorithms for hard network
optimization problems.
\iffalse (Repeats what is in the prev para)
This proposal encompasses several challenging
aspects of mission-critical networked systems: (a) there is
considerable uncertainty in the inputs and the network environment
under which the systems operate; (b) the inputs as well as the
underlying network may, in fact, be under adversarial control; (c) the
algorithms running these networked systems need to be
fully-distributed.
\fi
At a high level, the problems we plan to study in this project can be
divided into two categories: {\bf \em network design}\/ and {\bf \em
information flow}.

\BfPara{Network Design} Network design concerns the construction of
overlay network structures that form the foundation for aggregation,
point-to-point routing, broadcast, multicast and other critical
network functions.  The deployment of mission-critical military
systems requires the ability to construct and maintain such
large-scale network structures that will enable secure and reliable
communication and operation in a highly dynamic and distributed
environment.  Research in network design has been at the forefront of
major advances in approximation algorithms.  We believe that
theoretical foundations of distributed online network design are
essential to achieve major advances in this area.  Within network
design, we will study the general problem of {\em survivable network
  design}, with a focus on the important special case of {\em
  aggregation trees}.

\begin{itemize}
\item
{\sl Survivable network design:} In the {\em survivable network design
  problem}, we are given a graph $G = (V,E)$ with edge-costs, and
edge-connectivity requirements $r_{ij} \in Z_{\ge 0}$ for every pair
of vertices $i, j \in V$, and need to find an (approximately)
minimum-cost network that provides the required connectivity.  This is
one of the most fundamental problems in network design that
generalizes several graph-theoretic optimization problems including
shortest paths, spanning and Steiner trees.  The edge-connectivity
requirements capture the need for increased reliability in
mission-critical systems; furthermore, the general statement of the
problem allows one to develop algorithmic paradigms that may have
broad applicability.

\item
{\sl Aggregation trees:} A special case of survivable network design
is the fundamental problem of constructing a tree that aggregates
information from important agents in a distributed network to a
central point (HQ). The agents are modeled as nodes in a graph, whose
edges capture the communication links between the agents.  Suppose the
primary task of the network is to detect and monitor specific targets
whose location the agents are unaware of a priori.  As the agents
explore their local terrain in detail they may become aware of the
presence of valuable targets. This converts such a node into a {\em
  terminal} node that must now communicate with the HQ node henceforth
called the {\em root}.  This problem is a variant of the classical
minimum Steiner tree problem.  Though Steiner tree and their variants
have been extensively studied, there are no distributed algorithms for
computing near-optimal Steiner trees in dynamic or uncertain
environments.  We propose new distributed methods for constructing
near-optimal aggregation trees in stochastic and adversarial
scenarios.
\end{itemize}

\BfPara{Information Flow}
Information flow concerns the development of decentralized algorithms
that route higher-level application information over networks.  Within
information flow, we will study distributed algorithms for
constructing and maintaining routing tables, and gossip-based
information dissemination in highly dynamic networks.

\begin{itemize}
\item
{\sl Minimizing congestion in dynamic networks:} The capacity of links
in wireless (ad hoc and sensor) networks often changes due to fading
and multi-path effects.  Further, the mobility of the nodes themselves
leads to changes in the network topology.  A major challenge in such a
dynamic environment is to compute and maintain bottleneck-free routes
for information flow.  There are two main sources of difficulty:
first, the routes must satisfy certain degree constraints to for
keeping routing information within manageable limits~\cite{xxx};
second, each node only has knowledge of the state of links to which
they are connected, while link capacities can be dynamically varying.
Given a traffic demand matrix over the set of nodes, our goal is to
devise a distributed algorithm that computes and maintains {\em
  congestion minimizing degree-bounded flows}\/ between all
source-destination pairs.  The algorithm must be rapidly convergent
and must utilize minimal control information to avoid congesting the
links that must be conserved for data transport.

\item
{\sl Information dissemination in adversarially dynamic networks} The
dynamics of networks arising in military and critical infrastructure
settings are not only extensive, but can also be influenced by
external adversaries.  A natural model is that of an adversarial {\em
  time-evolving network}: in step $t$, the network $G_t = (V_t, E_t)$
is an arbitrary graph over the set $V_t$ of $n$ nodes, where the edge
sets $E_t$ are determined by an adversary.  We propose to develop
purely local lightweight algorithms for fundamental information
dissemination tasks in highly dynamic networks.  Our starting point
will be the basic {\em $k$-broadcast problem} in $k$ of the $n$ nodes
each have a message and would like to disseminate them to every node
in the network.  A fundamental open problem is the following: Can the
$k$-broadcast problem be solved on an dynamic, always-connected,
$n$-node network be solved in $O(n + k)$ steps?
\end{itemize}

\subsection{Future Naval Relevance}

\subsection{Technical Approaches and Recent Related Work}
The optimization problems raised in this proposal pose two major
challenges.  First, how do we deal with uncertainty, whether it is in
the node pairs being connected in the survivable network design
problem, the location of targets in the aggregation tree problem, or
the dynamic network topology in the information flow problems?
Second, how do we ensure that our algorithms can be implemented with
little coordination among the network nodes?

We plan to attack the issue of uncertainty by developing algorithms
for a spectrum of models spanning from {\em stochastic}, where the
unknown online inputs are assumed to be drawn from some known
probability distribution, to {\em adversarial}, where the inputs or
network dynamics are completely under adversarial control.  For the
design of fully-distributed algorithms, we propose to study new {\em
  distributed methods}\/ for constructing network decompositions that
are at the heart of many effective network design algorithms.  We will
also focus on gossip-style algorithms for information flow, which are
inherently decentralized, and have the promise of achieving
near-optimality in highly dynamic environments.  We now elaborate on
the specific technical approaches for the problems listed in
Section~\ref{sec:goals}.

\subsection{Network Design}
\subsubsection{Survivable Network Design}
Since the SNDP is NP-hard (it contains the Steiner tree problem as a special case), it has been widely studied from the viewpoint of approximation algorithms (See e.g.,~\cite{GK11} for a survey of results). These problems were
one of the earliest applications of the primal-dual method in this area
which led, over a sequence of papers, to the development of an $O(\log
r_{\max})$-approximation algorithm~\cite{GGPSTW94}.  Subsequently, one of the first uses of iterative rounding in approximation algorithms
led to a $2$-approximation for this problem (and for the general problem
of network design with weakly-supermodular functions)~\cite{Jain01}.

In~\cite{gkr10}, we studied these problems in the
\emph{online} setting: we are given a graph with edge-costs, and an
upper bound $r_{\max}$ on the connectivity demand. A sequence of vertex pairs $\{i,j\} \in V
\times V$ is presented to us over time, each with some edge-connectivity
demand $r_{ij}$---at this point we may need to buy some edges to ensure
that all the edges bought by the algorithm provide
an edge-connectivity of  $r_{ij}$ between vertices $i$ and $j$.  The goal is to remain
competitive with the optimal offline solution of the current demand set.
Work in this online
setting has mainly focused only on one-connected problems such as
Steiner and generalized Steiner trees~\cite{qw11}.
No online algorithms were previously known
for this problem even for the online rooted $2$-connectivity problem (i.e., for
the case where all the vertex pairs share a root vertex $r$ and the
connectivity requirement is~$2$ for all pairs) while we noted
a lower bound of $\Omega(\min\{|D|, \log n\})$ on the competitive ratio for this special case, where $D$ is the set of terminal pairs given to the
algorithm. This is in contrast to the case of online $1$-connectivity (i.e., online Steiner forest) where the best online algorithm is $\Theta(\log |D|)$-competitive \cite{BCsteiner97}.
Our main result for the edge-connected survivable network design problem is an $O(r_{\max} \log^3 n)$-competitive randomized online algorithm against oblivious adversaries.

Our algorithms use the standard embeddings of graphs into random subtrees (i.e., into \emph{singly    connected} subgraphs) as an intermediate step to get algorithms for higher connectivity. As a consequence of using these random embeddings, our algorithms are competitive only against oblivious adversaries. A natural extension that is still open is whether out methods can be extended to work against adaptive adversaries.

Furthermore, our algorithms relying on tree embeddings and inherently centralized.  
We propose to study decentralized algorithms for survivable network design in an online
setting where the underlying graph and costs are static while the
connectivity requirements change online. 

Finally, our algorithms are all centralized, and it is unclear how to extend even the one-connected online algorithms for Steiner trees to the distributed setting.

\Research{
\label{prob:sndp}
Devise distributed online algorithms for the Survivable Network Design Problem with good competitive ratios against adaptive adversaries.}

Another direction to extend the SNDP is the
stochastic case when the instance is drawn according to a probability
distribution. In ~\cite{gkr10}, we considered the case when we have a product
distribution: for each pair $i,j$ of vertices we are given a probability $p_{ij}$,
and are guaranteed that \emph{tomorrow} each pair will flip their coins
independently, and if the coin turns up heads, they would demand
$k$-connectivity. We can buy some edges today at cost $c(\cdot)$, but if
we wait for the actual set $D$, the edges will cost $\lambda c(\cdot)$
tomorrow, for a pre-specified inflation parameter $\lambda \geq 1$; the goal
is to minimize the sum of the cost of edges bought today and the
expected cost of augmentation edges bought tomorrow (at the inflated price).
In that work, we assumed for simplicity that all pairs have the the same
connectivity requirement of $k$. A simple generalization would allow each arriving pair to specify even the connectivity to be any value in $\{0,1\ldots,k\}$ according to a pre-defined probability distribution -- we'll now have values $p_{ij}(\kappa)$ for $\kappa \in \{0,1\ldots,k\}$, summing to one. We propose to extend our previous work to this more interesting setting that allows for more general stochastic specifications of connectivity requirements.

\subsubsection{Aggregation trees}
Consider the problem of building an aggregation tree for tracking
targets.  In a stochastic model of uncertainty, we may view each
target as an object that is detected by one of the various agents
according to a previously determined probability distribution.  In
order to build a backbone network of the graph for coordination and
control, we need to determine a subtree of the graph containing the
root such that given the probability distribution of target
materialization, the subtree maximizes the expected number of targets
that can be covered by the agent nodes in the tree. If this backbone
tree is a spanning tree, then all targets (each of which materializes
at any one of the nodes) will be covered. At the other end, if no
backbone or tree edges can be built a priori, the root can only detect
targets that materialize in its vicinity. We thus have a trade-off
between the allowed size of the Steiner tree and its expected target
coverage that we propose to study in detail.

We call this problem the \noindent{\bf \em Target Maximizing Rooted
Steiner Tree} problem.  We are given an undirected graph $G =
(V,E)$, a root node $r$, and a set of targets $T_i, i = 1 \ldots
t$. Each target $T_i$ is a probability distribution $p_i$ over $V$
where $p_i(v)$ is the probability that the target $i$ is at node $v$
(Note that $p_i(v) \geq 0 \ \forall v$ and $\sum_{v \in V} p_i(v) = 1
\ \forall i$). Assume that the distributions $p_i$ and $p_j$ for $i
\neq j$ are independent of each other. Given a budget $S$ on the size, our
goal is to find a tree $F$ containing the root of at most $S$ edges
such that the expected size of the targets that materialize in the
node set of the tree is maximized.

This problem can be generalized in various ways. First, the size bound
$S$ can be extended to take into account distances between nodes or
the relative costs of establishing connection between these pairs of
agents. Second, the objective function can be changed to reflect other
notions of reward such as the probability of detecting at least $N$
targets for a given $N$. Third, the independence assumption on target
materialization can be relaxed; this would not make any difference for
the expected target size function but is interesting for other
objectives. Fourth, the tree construction process can be modeled as
the vehicle tour of a control packet in the ad hoc network, leading to
vehicle routing problems that try to minimize expected size to cover
all targets say. Fifth, the notion of stochastic uncertainty can be
replaced by the requirement that the method is robust towards the
worst of many possible scenarios of target materialization. We
investigate one particularly strong notion of robustness for this
problem in the next subsection.

\Research{
\label{prob:tmrst}
Devise offline approximation algorithms for the Target Maximizing Rooted
Steiner Tree problem with good constant performance ratio. Extend the techniques to the various generalizations and to the robust and distributed settings.}

As stated, the problem is closely related to the $k$-MST problem that two of us introduced in an early paper~\cite{RSMRR94}, and for which we eventually designed the first constant-factor approximation algorithm~\cite{BRV99}. In this problem we are given an undirected edge-weighted graph with a root and an integer $k$, and the goal is to find a minimum-weight spanning tree connecting the root with at least $k$ other nodes. Our target maximizing version is close to the complementary version of the problem where we are given a bound on the cost of the tree and the goal is to maximize the number of nodes covered. This is also related to the {\em Orienteering Problem} where the object collecting the nodes is an orienteering path rather than a tree, and for which constant factor approximations are known~\cite{BCKLMM03}. We have recently also devised approximation algorithms for stochastic versions of the orienteering problem~\cite{GKNR12} but the stochasticity is related to delays at the nodes to fetch the rewards rather than the location of targets on the visited nodes. These are all interesting approaches that we propose to investigate to solve the target maximizing problems we propose. Extending these centralized methods to work in a distributed setting is also a challenging direction we hope to purse.

\iffalse
RELATED WORK: karger-minkoff maybecast tree, gupta-nagarajan-ravi
paper on adaptive TSP (icalp 10), and on stochastic vrp (OR, 2012) and
refs in this for previous papers on TSP with independent demands on
nodes (Bertsimas cycle heuristic). Also the basic expected target max
is like orienteering with size bound on tree, and hence related to
k-MST. Also related to garg-gupta-leonardi-sankowsi (SODA 08) which
relates to Univ TSP.
\fi

\subsubsection{Universal approximations for Steiner trees}
We also plan to study aggregation trees under an adversarial model of
uncertainty.  We have introduced the framework of {\em universal
  approximations}\/ that provides a robust notion of quality with
respect to {\em any} online sequence of
arrivals~\cite{jia+lnrs:universal}.  Universality is a framework for
dealing with uncertainty by guaranteeing a certain quality of goodness
for all possible completions of the partial information set.
Formally, an instance of the Universal Steiner Tree (\ust) problem is
a pair $\langle G , r\rangle$ where $G = (V,E)$ is a weighted
undirected graph, and $r$ is a distinguished vertex in $V$ that we
refer to as the {\em root}.  For any spanning tree $T$ of $G$, define
the {\em stretch}\/ of $T$ as $\max_{S \subseteq V}\cost{T_{S\cup
    r}}/\cost{\OptSt{S\cup r}}$, where $\OptSt{X}$ is an optimal
Steiner tree connecting the vertices in $X$.  The goal of the
\ust\ problem is to determine a spanning tree with minimum stretch.

Similar to the \ust\ problem, we can define universal versions of
Traveling Salesperson Problem (\utsp), group Steiner tree, group TSP,
and generalized Steiner network problems.  An instance of \utsp\ is a
metric space $(V,d)$.  For any cycle (tour) $\TR$ containing all the
vertices in $V$ and a subset $S$ of $V$, let $\TR_S$ denote the unique
cycle over $S$ in which the ordering of vertices in $S$ is consistent
with their ordering in $C$.  The stretch of $\TR$ is defined as
$\max_{S\subseteq V} \cost{\TR_{S}}/\cost{\Optr{S}}$, where $\Optr{S}$
denotes the minimum cost tour on set $S$. The \utsp\ is to find a tour
on $V$ with minimum stretch.

\Research{
\label{prob:ust}
What are the best stretch achievable for \ust, \utsp, and universal
variants of the group Steiner tree, group TSP, and generalized Steiner
network problems?}

The notion of universality is captured by the complexity class
$\Sigma_2^P$.  Previous work has shown that for any metric space,
there exist Steiner trees that have universal poylogarithmic
approximation ratios: our prior results~\cite{jia+lnrs:universal}, and
the improved results of $O(\log^2 n)$~\cite{gupta+hr:oblivious} that
introduce the notion of oblivious network design.  Though these
results reveal new onsights to the structural properties of Steiner
trees and metrics, a major drawback is that {\em they do not apply to
  arbitrary graphs}.

In very recent work, we have made some progress on this
front~\cite{busch+drrs:ust}, and have identified a promising approach
to attack the problem.  One of the major challenges in constructing a
universal Steiner tree is that any tree will be forced to place some
vertices far apart in tree, even though they may be ``nearby''
according to the underlying graph distances.  As a result, the
resulting spanning tree may perform poorly on a subset that includes
this set and some other carefully chosen vertices.  To address this
challenge, we introduce the notion of a {\em hierarchy of graph
  partitions}, each of which guarantees small strong cluster diameter
and bounded local neighbourhood intersections.  We have shown that the
such a suitable hierarchy of graph partitions is essentially both
sufficient and necessary for constructing low-stretch universal
Steiner trees.  For metric spaces -- i.e., weighted complete graphs
satisfying triangle inequality -- such a hierarchy can be constructed
using the seminal work of Awerbuch and Peleg on sparse
partitions~\cite{awerbuch+p:partition,peleg:distributeBook}.

It is a major open problem, however, to build such hierarchies for
arbitrary graphs.  We have made preliminary progress by presenting
partition hierarchies for general graphs that yield a $2^{O(\sqrt{\log
    n})}$-stretch \ust\ for general graphs, and partition hierarchies
for minor-free graphs that yield a polylogarithmic-stretch \ust\ for
minor-free graphs.  Furthermore, all of the solutions proposed thus
far are centralized.  We propose to develop distributed algorithms for
constructing sparse partition hierarchies.  In an earlier work on
aggregation trees for sensor network applications, we showed that this
is achievable for the highly specialized case of grid
graphs~\cite{jia+nrs:gist}.


\subsection{Information Flow}
Our proposed work on decentralized network design will ensure that the
underlying communication network connecting the agent nodes satisfies
mission-critical connectivity properties.  The network structures thus
designed are very likely to change with time due to the mobility of
the nodes, failures in the communication links, as well as adversarial
attacks on the network.  The second major component of this proposal
to develop distributed algorithms for efficient and effective flow of
information over such dynamic networks.  Our proposed work on network
design allows us to make assumptions about minimal connectivity of the
underlying network; however, to guarantee high-throughput information
flow for a dynamic distributed environment, we need to control
congestion, and focus on local lightweight algorithms that maintain
limited state.  In this project, we will study two important
information flow problems: congestion-minimizing degree-bounded flows,
and information dissemination under adversarial dynamics.

\subsubsection{Congestion-minimizing degree-bounded flows in dynamic
  networks} In earlier work \cite{awerbuch+l:flow}, distributed
algorithms for finding congestion minimizing flows in certain dynamic
networks were presented.  However the routes undertaken by these flows
were unconstrained and expensive to maintain.  We plan to build on our
past work on efficient algorithms for converting optimal flows into
near-optimal degree-bounded
flows~\cite{chen+klrsv:confluent,chen+smr:confluent}; our current
algorithms involve some refinement techniques that require global
knowledge.  We believe that the local balancing approach
of~\cite{awerbuch+l:flow} in conjunction with our flow refinement
techniques will enable us to maintain near-optimal degree-bounded
flows in dynamic networks.  We also plan to consider generalizations
of the problem incorporating link errors and coding techniques.

This problem can be generalized in a multiplicity of ways. Rather than
one commodity one can consider multiple commodities, each with their
own target or destination nodes. Second, one can consider the
availability of end-to-end capacities assuming duplex
connections. Third, one can assume different forms of error that
affect link capacities and the use of techniques such as network
coding. Finally, one can also consider integrating the route discovery
problem with the problem of optimizing the flows.


\Research{
\label{prob:cmdf}
Design distributed algorithms for congestion-minimizing degree-bounded flows in dynamic multicommodity networks.
}

\subsubsection{Information dissemination in adversarial networks}
We will conduct a comprehensive study of fully distributed algorithms
for information dissemination in dynamic networks controlled by
adversaries.  In such highly dynamic scenarios, the algorithms need to
be lightweight and maintain very limited state since it is impossible
to coordinate or reach significant
consensus~\cite{ruppert+l:impossibility}.  We will begin by studying
the basic $k$-broadcast problem.  Recall that in the $k$-broadcast
problem, $k$ of the $n$ nodes each have a message that need to be
disseminated to every node in the network.  Suppose that the nodes are
synchronized and in each step, each node can broadcast the equivalent
of a bounded number of tokens to its neighbors~\cite{kuhn+lo:dynamic}.
What is the minimum number of steps needed to complete the
dissemination?  If the network is completely static and connected,
then a local token-forwarding process on a spanning tree of the
network can accomplish the task in $O(n + k)$ steps, independent of
the structure of the network.  In a dynamic network as in the above
model however, the problem is much more challenging.

\Research{
\label{prob:broadcast}
Is there a distributed algorithm that completes any $k$-broadcast
instance in $O(n + k)$ steps on any adversarially dynamic $n$-node
network? }

In recent work~\cite{dutta+prs:dynamic}, we studied the class of {\em
  forwarding}\/ algorithms that do not manipulate tokens in any way
other than copying, storing, and forwarding them.  We show that {\em
  any}\/ forwarding algorithm will take $\Omega(nk/\log(n))$ steps to
complete $k$-broadcast, thus resolving an open problem
of~\cite{kuhn+lo:dynamic}.  Given that almost any local greedy
forwarding procedure completes $k$-broadcast in $O(nk)$ steps in any
dynamic network, our lower bound essentially captures the severe
limitations imposed by highly adversarial network dynamics.

A natural and attractive alternative to forwarding algorithms is to
use coding (either end-to-end~\cite{Byers02adigital,Shok06} or
network~\cite{ahlswede+cly:coding}).  Recent
work~\cite{haeupler:gossip,haeupler+k:dynamic} has shown that
information spreading based on network coding can solve $k$-broadcast
in $O(n+k)$ steps, assuming the sizes of the messages are $\Omega(n
\log n)$ bits.  While this message size lower bound is prohibitively
large and impractical (since it scales with the size of the network),
our lower bound~\cite{dutta+prs:dynamic} together with this upper
bound establish, in theory, a fundamental gap between flow-based and
coding-based dissemination procedures.

One approach we plan to pursue is to consider a hybrid
forwarding-coding algorithm in which nodes exchange information in the
symmetric difference of what they currently hold, which can be done in
$O(\log n)$ rounds of communication using
fingerprinting~\cite{mitzenmacher-2005-fastmixing}.  We have shown
that if the entropy of the initial distribution of information
is high, then convergence to full dissemination is rapid.  We will
also consider weaker notions of the adversary (e.g., offline or
oblivious), which model real-world settings where the adversary has
significant control but is not congnizant of all the network actions.
The offline dissemination problem can be reduced to the problem of
packing Steiner trees in directed
graphs~\cite{cheriyan+s:steiner,dutta+prs:dynamic}, and thus has deep
connections with the long-standing open problems of approximating
directed Steiner
trees~\cite{charikar+ccdgg:steiner,halperin+k:steiner,zosin+k:steiner}
and bounding the network coding advantage in multicast over directed
networks~\cite{agarwal+c:coding,sanders+et:flow}.

We expect to quantify our results in terms of relevant parameters
including locality of the dynamics, conductance/expansion of the
evolving network, initial entropy of the information distribution, and
knowledge available to the dissemination algorithm.  
\iffalse***** Repeats last para
We have also
shown that the dissemination problem can be reduced to the problem of
packing Steiner trees in directed graphs, and thus has deep
connections with the long-standing open problems of approximating
directed Steiner trees~\cite{charikar+ccdgg:steiner} and bounding the
network coding advantage in multicast over directed networks.
\fi

\Research{
\label{prob:dynamic_general}
Design and analyze fully distributed gossip-based algorithms for
consensus and aggregation, and computing separable functions. }


